\documentclass{article}

\usepackage{fancyhdr}
\usepackage{extramarks}
\usepackage{amsmath}
\usepackage{amsthm}
\usepackage{amssymb}
\usepackage{amsfonts}
\usepackage{tikz}
\usepackage[plain]{algorithm}
\usepackage{algpseudocode}
\usepackage{nameref}
\usepackage{cite}
\usepackage{tikz-cd}
\usepackage{mathrsfs}
\usepackage{tikz}
\newcommand*\circled[1]{\tikz[baseline=(char.base)]{
            \node[shape=circle,draw,inner sep=2pt] (char) {#1};}}

\usetikzlibrary{automata,positioning}


\topmargin=-0.45in
\evensidemargin=0in
\oddsidemargin=0in
\textwidth=6.5in
\textheight=9.0in
\headsep=0.25in

\linespread{1.1}

\pagestyle{fancy}
\chead{\hmwkTitle}
\lhead{\hmwkAuthorName}
\rhead{\hmwkClass}
\cfoot{\thepage}

\renewcommand\headrulewidth{0.4pt}
\renewcommand\footrulewidth{0.4pt}
\newcommand{\sur}[1]{\ensuremath{^{\textrm{#1}}}}
\newcommand{\sous}[1]{\ensuremath{_{\textrm{#1}}}}
\newcommand{\Hom}{\text{Hom}}
\newcommand{\Tor}{\text{Tor}}
\newcommand{\Ext}{\text{Ext}}
\newcommand{\bb}[1]{\mathbb{#1}}
\newcommand{\fk}[1]{\mathfrak{#1}}
\newcommand{\iso}{\cong}

\setlength\parindent{0pt}

%c
% Create Problem Sections
%

\newtheorem{lemma}{Lemma}
\newtheorem{exercise}{Exercise}
%
% Homework Details
%   - Title
%   - Due date
%   - Class
%   - Section/Time
%   - Instructor
%   - Author
%

\newcommand{\hmwkTitle}{Homework 3}
\newcommand{\hmwkDueDate}{Oct 23rd, 2019}
\newcommand{\hmwkClass}{Math 215A Commutative Algebra}
\newcommand{\hmwkClassInstructor}{Professor James Cameron}
\newcommand{\hmwkAuthorName}{\textbf{Anish Chedalavada}}

%
% Title Page
%

\title{
    \vspace{2in}
    \textmd{\textbf{\hmwkClass:\ \hmwkTitle}}\\
    \vspace{0.1in}
    \textmd{\hmwkDueDate} \\
    \vspace{0.2in}\large{\textit{\hmwkClassInstructor\  }}
    \vspace{2in}
}

\author{\hmwkAuthorName}
\date{}

\begin{document}
\maketitle
\newpage

\begin{exercise}
Show that being reduced is a local property, i.e. $R$ reduced $\iff R_{\fk{m}}$ reduced for all maximals $\iff R_{\fk{p}}$ reduced for all primes.  
\end{exercise}
\begin{proof}
  Suppose the localization at every prime ideal is reduced, this clearly implies the localization at every maximal ideal is reduced as this is strictly weaker. Now, suppose $R_{\fk{m}}$ is reduced for all $\fk{m}$. We have that, $R_{\fk{m}}\otimes_{R} \sqrt{0} \subset \sqrt{0}_{\fk{m}}$ where $\sqrt{0}_{\fk{m}}$ denotes the nilradical of the localization. It is clear to see this as multiplying a nilpotent element by a unit preserves nilpotency in a commutative ring. Thus, by assumption, $R_{\fk{m}}\otimes_{R} \sqrt{0}$ is 0, being contained in a $0$ module. Furthermore, this is true at all maximal ideals, and so as a module $R_{\fk{m}}\otimes_{R} \sqrt{0}$ is trivial, implying that as an $R$-module and thus an ideal, $\sqrt{0}$ is trivial. Now suppose that $\sqrt{0} \subset R$ is trivial. We have that at any localization $R_{\fk{p}}$, the nilradical must be generated by the image of elements in $\sqrt{\text{ker} f}$ for $f$ the localization map, as all ideals in the localization are generated by ideals in $R/\text{ker}f \subset R_{\fk{p}}$, and the result is clear in $R/ \text{ker}f$. Thus, it suffices to show that $\sqrt{\text{ker} f}$ is $\text{ker}f$ itself for arbitrary closed subsets excluding a prime. However, we know that $\text{ker}f = \{r \in R \ | \exists \ s\in S \text{ s.t. }sr=0\}$. Thus, for any element $a \in R$ s.t. $a^{n} \in \text{ker}f$, we have that $sa^{n} = 0$ some $s$, implying that $sa$ is nilpotent, and so $sa=0$ by triviality of the nilradical, implying $ a \in \text{ker}f$. Thus, the radical of ker$f$ is itself, and so the nilradical of $R_{\fk{p}}$ must be trivial. As $\fk{p}$ was arbitrary, this must be true at all primes, concluding the proof.
\end{proof}
\begin{exercise}
  Show that $k[x,y]/(x^{2}-xy)$ is not flat as a $k[x]-$module. 
\end{exercise}
\begin{proof}
  Consider the short exact sequence given by:
  \[
  \begin{tikzcd}
    0 \arrow[r] & k[x] \arrow[r,"x"] & k[x] \arrow[r] & k \arrow[r] & 0
  \end{tikzcd}
\]
Upon tensoring with $k[x,y]/(x^{2}-xy)$ this becomes the following right exact sequence:
  \[
  \begin{tikzcd}
     k[x,y]/(x^{2}-xy) \arrow[r,"x"] & k[x,y]/(x^{2}-xy) \arrow[r] & k[y]/(xy) \arrow[r] & 0
  \end{tikzcd}
\]
Where the last module is derived from the fact that the tensor product is right exact. It is clear, however, that the map
 \[
  \begin{tikzcd}
     k[x,y]/(x^{2}-xy) \arrow[r,"x"] & k[x,y]/(x^{2}-xy)
  \end{tikzcd}
\]
Has the element $(x - y)$ in its kernel, and is thus not injective. Therefore, this module is not flat.
\end{proof}


\begin{exercise}
1) Give a construction of pushouts in the category of rings, and show how this gives coproducts in rings when $A$ is chosen appropriately. \\  
2) Give an example to show that $Spec$ of a pushout is not generally sent to a pullback.
\end{exercise}
\begin{proof}
  1) Given any commuting diagram of the following form:
  \[
    \begin{tikzcd}
      A \arrow[r, "i_{1}"] \arrow[d, "i_{2}"] & B \arrow[d, "f"] \\
      C \arrow[r, "g"] & X
    \end{tikzcd}
  \]
We have that any map out of the cone $C \leftarrow A \rightarrow B$ into $X$ yields a unique set map from $C \times B$ the cartesian product (i.e. not the ring product) given by $h: (b,c) \mapsto f(b) \cdot g(c)$, and the set map $f$ factors as $f = h \iota_{B}$ for $\iota_{B} : B \hookrightarrow B \times C$ via $b \mapsto (b,1)$: this works similarly for $\iota_{C}$ and $g$. We have the following diag

  Note that the following properties are satisfied:
  \begin{itemize}
  \item $h$ is bilinear in both coordinates: $h(b_{1}+b_{2},c) = (f(b_{1})+f(b_{2})) g(c) = h(b_{1},c) + h(b_{2},c)$, and symmetrically for $C$.
  \item $h$ is $A$ bilinear by the commutativity of the diagram above: $A \to B \to X = A \to C \to X$, implying $A \to B \to B \times C \to X = A \to C \to B \times C \to X$, and so $h(i_{1}(a), 1) = h(1, i_{2}(a))$. This gives us one better, as $h(i_{1}(a)b, c) = f(b) \cdot fi_{1}(a) \cdot g(c) = f(b) \cdot gi_{2}(a) \cdot g(c) = h(b, i_{2}(a))$. 
  \end{itemize}
  Thus, $X$ receives a unique map from the tensor product as $A$ algebras, $B \otimes_{A} C \to X$ such that the following diagram commutes:
  \[
    \begin{tikzcd}
A \arrow[r, "i_{1}"] \arrow[d, "i_2"]          & B \arrow[d, "\iota_1"] \arrow[rdd, bend left] &   \\
C \arrow[r, "\iota_2"] \arrow[rrd, bend right] & B \otimes_A C \arrow[rd, dotted, "\exists \ !"]              &   \\
                                               &                                               & X
\end{tikzcd}
\]
And thus by universal property $B \otimes_{A} C$ must be the pushout of this diagram. We may derive coproducts from this construction by noting that any two objects receive unique maps from the initial object in \textbf{CRing}, which is $\bb{Z}$, and so the coproduct is in fact the pushout of $B \leftarrow \bb{Z} \rightarrow C$, which from above must be $B \otimes_{\bb{Z}}C$.  \\

2) Consider the following diagram:
    \[
    \begin{tikzcd}
      \bb{C} \arrow[r] \arrow[d] & \bb{C}[t] \arrow[d] \\
      \bb{C}[t^{-1}] \arrow[r] & \bb{C}[t^{-1}] \otimes_{\bb{C}}\bb{C}[t] \iso \bb{C}[t,t^{-1}]
    \end{tikzcd}
  \]
  We have that the $Spec$ of $Spec(\bb{C}[t^{-1}]) \iso Spec(\bb{C}[t]) = \bb{A}^{1}$ as \underline{topological spaces}. We also know that $\bb{C}[t,t^{-1}]$ is the $Spec$ of a localization, is $Spec(\bb{C}[t]) \setminus (t) \iso \bb{A}^{1}\setminus \{0\}$.Thus, under $Spec$, this is sent to the diagram:
  \[
        \begin{tikzcd}
      \bb{A}^{1}\setminus \{0\} \arrow[r] \arrow[d]& \bb{A}^{1} \arrow[d] \\
      \bb{A}^{1} \arrow[r] & *
    \end{tikzcd}
  \]
  In \textbf{Top} the pullback of the diagram $\bb{A}^{1} \rightarrow * \leftarrow \bb{A}^{1}$ is $\bb{A}^{1}$ itself mapping homeomorphically onto each space, as there is only one unique map to the point space. Thus, $Spec$ does not necessarily send pushouts to pullbacks.  
\end{proof}

\begin{exercise}
  1) Show that if M and N finitely generated modules over a local ring R then $M \otimes_{R}N =0$ if and only if either M or N is zero. \\
  2) Show the following: \\
  a) If $0\to L \to M \to N \to 0$ is exact then supp $M =$ supp $L \ \cup$ supp $N$. \\
  b) If $M, N$ are finitely generated then supp sends the tensor product to an intersection.
\end{exercise}
1) By Nakayama's lemma, we have that a module $L$ over a local ring if $L/\fk{m}$ is zero for $\fk{m}$ the maximal ideal. Thus, we have that $R/\fk{m} \otimes_{R}(M \otimes_{R} N) \iso (R/\fk{m} \otimes_{R/\fk{m}} R/\fk{m})\otimes_{R}(M \otimes_{R} N) \iso (M \otimes_{R} R/\fk{m}) \otimes_{R/\fk{m}}(R/\fk{m} \otimes_{R} N) \iso M/\fk{m} \otimes_{R/\fk{m}} N/\fk{m}$ heavily invoking the associativity and commutativity of tensor products over commutative rings. However, $M/\fk{m} \otimes_{R/\fk{m}}N/\fk{m}$ is a tensor product of finite dimensional vector spaces. Let $m$ the dimension of $M/\fk{m}$ and $n$ the dimension of $N/\fk{m}$, with respective bases $\mu_{i}$, $\nu_{j}$. Consider the vector space $(R/\fk{m})^{mn}$ with the basis $\{e_{ij}\}$ for $0\leq\leq m$ and $0 \leq j \leq n$. Define maps $(R/\fk{m})^{mn} \to  M/\fk{m} \otimes_{R/\fk{m}}N/\fk{m}$ via $e_{ij} \mapsto \mu_{i} \otimes \nu_{j}$ and the inverse defined in the obvious fashion. This implies  $M/\fk{m} \otimes_{R/\fk{m}}N/\fk{m}$ has dimension $mn$, which is 0 if and only if either $m = 0$ or $n = 0$ which, from work above using Nakayama's lemma, implies $M = 0$ or $N = 0$, yielding the result using the same logic. \\
2) a) We use the flatness of the localization functor to show this claim: We have that supp $L \subset$ supp $M$, as for any prime in supp $L$ we have that $0 \to L_{\fk{p}} \to M_{\fk{p}} \to N_{\fk{p}} \to 0$ is exact and $L_{\fk{p}}$ is nonzero and injects into $M_{\fk{p}}$, which must thus also be nonzero. Symmetric logic yields that supp $N \subset M$. Finally, suppose $\fk{p} \notin$ supp $M \ \cup$ supp $N$. Then $0 \to 0 \to M_{\fk{p}} \to 0 \to 0$ is exact, implying $M_{\fk{p}} = 0$ and so it is not in supp $M$. This yields the claim. \\
b) Using part 1) and the methods of the proof, we have that $(M \otimes_{R} N)_{\fk{p}} \iso M_{\fk{p}} \otimes_{R} N_{\fk{p}} = 0 \iff M_{\fk{p}} = 0$ or $N_{\fk{p}} = 0$. Thus, $(M \otimes_{R} N)_{\fk{p}} \neq 0 \iff M_{\fk{p}}, N_{\fk{p}} \neq 0 \iff \fk{p} \in$ supp $M \ \cap$ supp $N$. 
\end{document}
