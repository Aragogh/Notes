\documentclass{article}

\usepackage{fancyhdr}
\usepackage{extramarks}
\usepackage{amsmath}
\usepackage{amsthm}
\usepackage{amssymb}
\usepackage{amsfonts}
\usepackage{tikz}
\usepackage[plain]{algorithm}
\usepackage{algpseudocode}
\usepackage{nameref}
\usepackage{cite}
\usepackage{tikz-cd}
\usepackage{mathrsfs}

\usetikzlibrary{automata,positioning}


\topmargin=-0.45in
\evensidemargin=0in
\oddsidemargin=0in
\textwidth=6.5in
\textheight=9.0in
\headsep=0.25in

\linespread{1.1}

\pagestyle{fancy}
\chead{\hmwkTitle}
\lhead{\hmwkAuthorName}
\rhead{\hmwkClass}
\cfoot{\thepage}

\renewcommand\headrulewidth{0.4pt}
\renewcommand\footrulewidth{0.4pt}
\newcommand{\sur}[1]{\ensuremath{^{\textrm{#1}}}}
\newcommand{\sous}[1]{\ensuremath{_{\textrm{#1}}}}
\newcommand{\Hom}{\text{Hom}}
\newcommand{\st}{\text \ \text{s.t.} \hspace{0.025in} }
\newcommand{\Claim}[1]{\vspace{0.05in} \\ \underline{\textbf{Claim:}} #1 \vspace{0.05in} \\}

\setlength\parindent{0pt}

%c
% Create Problem Sections
%

\newtheorem{lemma}{Lemma}
\newtheorem{exercise}{Exercise}
%
% Homework Details
%   - Title
%   - Due date
%   - Class
%   - Section/Time
%   - Instructor
%   - Author
%

\newcommand{\hmwkTitle}{Homework 2}
\newcommand{\hmwkDueDate}{Jan 18th, 2019}
\newcommand{\hmwkClass}{Math 225B Differential Geometry}
\newcommand{\hmwkClassInstructor}{Professor Peter Petersen}
\newcommand{\hmwkAuthorName}{\textbf{Anish Chedalavada}}

%
% Title Page
%

\title{
    \vspace{2in}
    \textmd{\textbf{\hmwkClass:\ \hmwkTitle}}\\
    \vspace{0.1in}
    \textmd{\hmwkDueDate} \\
    \vspace{0.2in}\large{\textit{\hmwkClassInstructor\  }}
    \vspace{2in}
}

\author{\hmwkAuthorName}
\date{}

\begin{document}

\maketitle

\newpage

\begin{exercise}
  Show the following: \\
  a) For any bundle $\pi: E \to B$, the map $s:B \to E$ with $s: p \mapsto 0 \in \pi^{-1}(p)$ is a section. \\
  b) Show that an n-plane bundle is trivial if and only if there exist everywhere linearly independent sections global $s_{1},...,s_{n}$.\\
  c) Show that locally every n-plane bundle has n linearly independent sections.
\end{exercise}

\begin{proof}
  a) It suffices to show that $s$ is continuous, as the composition with the projection is clearly the identity. We have that a base of open sets for a vector bundle locally corresponds with the product topology. Thus, the preimage of any open set in $E$ is determined by local preimages in $U_{i}\times \mathbb{R^{k}}$ for $U_{i}$ a cover of the manifold, and we have that the preimage of any open set $O \subset U_{i} \times \mathbb{R^{k}}$ is $O \cap U_{i} \times 0$ which is open in the relative topology on $U_{i}$, coinciding with the topology of the manifold and thus the section is continuous. \\
  b) One direction is clear, as if a bundle is trivial then we have the bundle is $M \times \mathbb{R}^{k}$ and we define $n$ linearly independent sections by $s_{i}: p \mapsto (p, (0,...,v_{i},...,0))$, which is continuous as it corresponds to the zero section followed by translation in the second coordinate. For the backwards direction, we define a homeomorphism from $M \times \mathbb{R}^{n}$ to $E$ given by $(p,x_{1},...,x_{n}) \mapsto (p, x_{1}s_{1}(p),...,x_{n}s_{n}(p))$. This map restricts to linear isomorphisms at each point as a linearly independent basis is sent to a linearly independent basis, and is thus bijective, continuous as every open set $E$ is the union of open sets of the form $\bigcup_{x_{i} \in V_{i}} U_{i}\times (s_{1}x_{1},...,s_{1}x_{n})$ as by linear independence, a local trivialization of every open set can be given with the basis of sections, and the preimage is $U_{i} \times V_{i} \subset  M \times \mathbb{R}^{n}$ ($V_{i}$ open by product topology on local trivialization. Clearly, open sets are mapped to open sets and this is a homeomorphism. \\
  c) For an $n-plane$ bundle, around every point we have a local trivialization given by $U \times \mathbb{R}^{k}$. Selecting any closed subset $K \subset U$ containing $p$, we define $n$ sections of $p$ by sending $p$ to $(p, w_{i})$ for $(w_{i})_{i=1,...,n}$ an orthonormal basis for $\pi_{2}(U \times \mathbb{R}^{k})$. This corresponds to the zero section followed by translation in the second coordinate and is clearly smooth. We define this to vanish on $M \setminus U$. This may be extended to a map on all of $M$ by the Tietze extension theorem (the map factors through a map to $1 \in \mathbb{R}$ on $K$ and to $0$ on $M\setminus U$. This yields a section of the manifold that locally maps to a basis element. Repeating this process for each basis vector yields $n$ distinct sections that are linearly independent on $p \in K^{o}$, i.e. satisfying the local condition. 
\end{proof}

\begin{exercise}
  If $g:\mathbb{R} \to \mathbb{R}$ is $C^{\infty}$ show that $g(x) = g(0) + g'(0)x + x^{2}h(x)$ for some $C^{\infty}$ function $h: \mathbb{R} \to \mathbb{R}$.
\end{exercise}
\begin{proof}
  Define $h(x) = \frac{g(x) - g(0) - g'(0)x}{x^{2}}$ for $x \neq 0$. At $0$ we evaluate the limit:
  \[
    \lim_{x \to 0}\frac{g(x)-g(0) - g'(0)}{x^{2}} = \lim_{{x \to 0}}\frac{g'(x) - g'(0)}{2x} = \frac{g''(0)}{2}
  \]
  By l'H\"opital's rule. Thus, this function is continuous. Furthermore, evaluating the derivative at 0 yields us:
  \[
    \lim_{x\to 0} \frac{g(x)-g(0)-g'(0)x - \frac{g''(0)}{2}}{x^{2}} = \lim_{x\to 0}\frac{g'(x) - g'(0)}{x} = \frac{g''(0)}{2}
  \]
  Which agrees on both sides of the limit. Thus, the function is clearly $C^{\infty}$ everywhere and differentiable at 0, and by similar logic as above we may show that higher derivatives of this function also exist for all $n$th derivatives. 
\end{proof}

\newpage

\begin{exercise}
  Show the following: \\
  a) Let p$_{0} \in S^{n-1}$ be the point $(0,...,1)$. For $n \geq 2$ define $f: SO(n) \to S^{n-1}$ by $f(A) = A(p_{0})$. Show that $f$ is continuous and open. Show that $f^{-1}(p_{0})$ is homeomorphic to $SO(n-1)$, and then show that $f^{-1}(p)$ is homeomorphic to $SO(n-1)$ for all $p \in S^{n-1}$. \\
  b) SO(1) is a point, so it is connected. Using part a) and induction on n show that SO(n) is connected for all n.\\
  c) Show that O(n) has exactly two components. 
\end{exercise}

\begin{proof}
  $f$ is clearly continuous, as under the normal metric topology on $\mathbb{R}^{n\times n}$ we have that for small perturbations in the entries of an $n \times n$ matrix that the resultant image points of a fixed vector are also perturbed by small amounts: thus, metric space continuity holds.
  \Claim{This map is open.}
  Let $A\in U$ open. We have some $\epsilon > 0$ s.t. $B_{\epsilon}(A) \subset U$ under the Euclidean metric. Thus, there is a $\delta > 0$ s.t. $\forall \ w \in B_{\delta}(v),  \exists \ B \in B_{\epsilon}(A) \st B(v) = w$. Thus, the map is open and we have the claim. The preimage of $p_{0}$ is all maps that fix the $n$th vector, which are of the form:
 $ \begin{pmatrix}
    A & 0 \\ 0 & 1
  \end{pmatrix} $
  with $A$ orthogonal $n-1$ dimensional square matrix with determinant 1; i.e. isomorphic to $SO(n-1)$ via the map $A \mapsto B$ by projecting onto the first $(n-1)^{2}$ coordinates. This is clearly continuous (it is a projection), injective as the last $2n-1$ coordinates are constant, and bijective as every $n-1$ dimension orthgonal matrix can be augmented to an $n$ dimensional one as above. Finally, it is a bijective continuous map from a compact space to a Hausdorff space, and is thus a homeomorphism. Finally, the preimage of any point $p$ is the left coset of the stabilizer of subgroup of $p_{0}$, which is homeomorphic to $SO(n-1)$. \vspace{0.1in} \\
  For part b), assume that $SO(n-1)$ is connected for some $n$. We have that $S^{n-1}$ is connected so the image of any two disjoint clopen sets must overlap at some point $p$. However, the preimage of $p$ is connected by assumption, yielding a contradiction. Thus, $SO(n)$ is connected. \\

  c) We know that $O(n)$ has matrices either of determinant positive or negative 1. $SO(n)$ is connected from b), and there exists a homeomorphism from one to the other via multiplication of $-1$ in the first column. They are disjoint, and thus there are two connected components. 
\end{proof}

\begin{exercise}
  a) Show that the matrix of the adjoint is the transpose matrix. \\
  b) Show that a symmetric matrix can be orthogonally diagonalized if we assume that it may be diagonalized. \\
  c) Show that a positive definite matrix is nonsingular. \\
  d) Show that $A^{T}\cdot A$ is positive semi-definite. \\
  e) Show that a positive semi-definite $A$ can be written as $A = B^{2}$ for some $B$. \\
  f) Prove polar decomposition. \\
  g) Show that $O_{1}$ and $P_{1}$ are continuous functions of $A$. \\
  h) Show that $GL(n,\mathbb{R}) \cong O(n, \mathbb{R}) \times P(n, \mathbb{R})$ 
\end{exercise}

\begin{proof}
  a) We have that if $\langle T^{*}v,w \rangle = \langle v,Tw \rangle$ then $T^{*}v^{T}\cdot w = v^{T} \cdot Tw$. Let $\{e_{i}\}_{i=1,...,n}$ be a basis for $\mathbb{R}^{n}$. For each basis element, we have that $e_{i}^{T} \cdot Ae_{j} = A^{T}e_{i}^{T}e_{j}$ as the first inner product represents the $i$th entry in the $j$th column, while the second one represents the $j$th entry in the $i$th column, and when $A_{ij}^{*} = A_{ji}$ we have that the adjoint must be the transpose. \\
  b) Selecting the first eigenvector $w_{1}$, we may generate an orthogonal basis (by Gram-Schmidt) with orthogonal change of basis matrix $O$ s.t.: \\
  \[
    OAO^{T} =
    \begin{pmatrix}
      \lambda_{1}  & ... & 0 \\
      \vdots & & \\
      0 & & \text{\Huge B} \\
    \end{pmatrix}
  \] \\ 
  And $B$ is clearly symmetric as $B^{T} = OA^{T}O^{T} = OAO^{T}|_{\mathbb{R}^{n-1}}$. Inductively proceeding yields the result. \\
  c) If it were singular then $\langle Tv, v \rangle = 0$ for $v \in \text{ker}T \setminus 0$. \\
  d) $\langle A^{T}\cdot A v, v\rangle = \langle Av, Av\rangle = ||Av||^{2} \geq 0$. \\
  e) $A$ is symmetric and thus orthogonally diagonalizable: positive semi-definite implies all eigenvalues are $\geq 0$, so we may define $B = \text{diag}\{\sqrt{\lambda_{1}},...,\sqrt{\lambda_{n}}\}$ in the diagonalized basis, positive semi-definite is preserved under orthogonal transformations and the diagonalized basis is orthgonal. \\
  f) We have that $A^{T}A = B^{2}$ for $B$ constructible as positive definite (can take all roots from e) positive, invertible implies nonzero eigenvalues). We have that $A = (A^{T})^{-1}B \cdot B}$. $(A^{T})^{-1}B \cdot ((A^{T})^{-1}B)^{T} = (A^{T})^{-1}A^{T}AA^{-1} = I$ and so $(A^{T})^{-1}B$ is orthogonal. Uniqueness follows as if $O_{2}^{T}O_{1} = P_{1}^{T}P_{2}$ then $O_{2}^{T}O_{1}$ is orthogonal and diagonalizable for all eigenvalues positive, which is only possible if all eigenvalues are 1 i.e. $O_{2}^{T}O_{1}$ is the identity. \\
g) If $A^{(n)} \to A$ is a convergent sequence then every subsequence of $A^{(n)}_{1}$ has a convergent subsequence by compactness of the orthogonal group, and this subsequence converges to the same limit $A_{2}$ by uniqueness of the polar decomposition. Given this fact, we have that $(A^{(n)}_{1})^{-1}A^{(n)} = A_{2}^{(n)}$, showing that $A_{2}^{(n)}$ must also converge to the limit $A_{2}$, and thus both functions are continuous. \\
h) We have a bijection from $GL(n, R) to O(n, \mathbb{R}) \times P(n, \mathbb{R})$ via polar decomposition. This is continuous in each coordinate from part $g$. Furthermore, the inverse, given by multiplication of the two coordinates, must be continuous as multiplication is continuous on a topological group.
\end{proof}

\begin{exercise}
  a) Show that a nonsingular linear transformation with positive determinant is homotopic to the identity map. \\
  b) Suppose $f:\mathbb{R}^{n}\to \mathbb{R}^{n}$ is $C^{\infty}$ and $f(0)=0$, $f(\mathbb{R}^{n}-0) \subset \mathbb{R}^{n}-0$ then $f: \mathbb{R}^{n}-0 \to \mathbb{R}^{n}-0$ is homotopic to $Df$. 
\end{exercise}
\begin{proof}
  a) Let $A:[0,1] \to GL(n,\mathbb{R})$ be continuous, and define $H:[0,1]\times \mathbb{R}^{n} \to \mathbb{R}^{n}$ by $H(x,t) = A(t)(x)$. Equipping a linear transformation with the operator norm (i.e. $||A(t)|| = \text{sup}_{||x||=1}||Ax||$), we have that $||A(t)x|| = ||A(t)||\cdot||x||$. Continuity then follows as $||A(t_{0})w - A(t_{1})v|| < ||A(t_{0})w - A(t_{1})w|| + ||A(t_{1})||\cdot||v-w|| < ||A(t_{0}) - A(t_{1})||.||w|| + ||A(t_{1})||\cdot ||v-w||$. We can independently impose restrictions on $t_{0}-t_{1}$ (by continuity) and $v-w$ s.t. the inequality above is less than $\epsilon$ for arbitrary $\epsilon > 0$, and thus this function is continuous. For a nonsingular linear transformation, we may define a homotopy to the identity as there is exists path between positive definite matrices by 31 h). \\

  b) We define a homotopy given by $H(x,t) = \frac{f(tx)}{t}$ for $0<t\leq 1$, and $Df(0)(x)$ for $t = 0$. We have that this is continuous at the origin (as it suffices to show continuity in the second coordinate at $t=0$, as the directional derivative is defined as the limit as $t \to 0$ for the difference quotient in $H(x,t)$ under the assumption that $f(0) = 0$.  
\end{proof}
\end{document}

 