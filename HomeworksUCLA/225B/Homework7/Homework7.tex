\documentclass{article}

\usepackage{fancyhdr}
\usepackage{extramarks}
\usepackage{amsmath}
\usepackage{amsthm}
\usepackage{amssymb}
\usepackage{amsfonts}
\usepackage{tikz}
\usepackage[plain]{algorithm}
\usepackage{algpseudocode}
\usepackage{nameref}
\usepackage{cite}
\usepackage{tikz-cd}
\usepackage{mathrsfs}

\usetikzlibrary{automata,positioning}


\topmargin=-0.45in
\evensidemargin=0in
\oddsidemargin=0in
\textwidth=6.5in
\textheight=9.0in
\headsep=0.25in

\linespread{1.1}

\pagestyle{fancy}
\chead{\hmwkTitle}
\lhead{\hmwkAuthorName}
\rhead{\hmwkClass}
\cfoot{\thepage}

\renewcommand\headrulewidth{0.4pt}
\renewcommand\footrulewidth{0.4pt}
\newcommand{\sur}[1]{\ensuremath{^{\textrm{#1}}}}
\newcommand{\sous}[1]{\ensuremath{_{\textrm{#1}}}}
\newcommand{\Hom}{\text{Hom}}
\newcommand{\st}{\text \ \text{s.t.} \hspace{0.025in} }
\newcommand{\Claim}[1]{\vspace{0.05in} \\ \underline{\textbf{Claim:}} #1 \vspace{0.05in} \\}

\setlength\parindent{0pt}

%c
% Create Problem Sections
%

\newtheorem{lemma}{Lemma}
\newtheorem{exercise}{Exercise}
%
% Homework Details
%   - Title
%   - Due date
%   - Class
%   - Section/Time
%   - Instructor
%   - Author
%

\newcommand{\hmwkTitle}{Homework 7}
\newcommand{\hmwkDueDate}{Feb 22nd, 2019}
\newcommand{\hmwkClass}{Math 225B Differential Geometry}
\newcommand{\hmwkClassInstructor}{Professor Peter Petersen}
\newcommand{\hmwkAuthorName}{\textbf{Anish Chedalavada}}

%
% Title Page
%

\title{
    \vspace{2in}
    \textmd{\textbf{\hmwkClass:\ \hmwkTitle}}\\
    \vspace{0.1in}
    \textmd{\hmwkDueDate} \\
    \vspace{0.2in}\large{\textit{\hmwkClassInstructor\  }}
    \vspace{2in}
}

\author{\hmwkAuthorName}
\date{}

\begin{document}

\maketitle

\newpage

\begin{exercise}[Problem 6]
\end{exercise}
\begin{proof}
  a) Consider the case of an arbitrary 2-form, expressible in a basis as: $\omega_{1}e_{1}\wedge e_{2} + \omega_{2} e_{2}\wedge e_{3} + \omega_{3} e_{3} \wedge e_{1}$ Given a solution to the equations
  \begin{align*}
    \omega_{1} = a_{2}b_{3} - a_{3}b_{2} \\
    \omega_{2} = a_{1}b_{3} - a_{3}b_{1} \\
    \omega_{3} = a_{1}b_{2} - a_{2}b_{1} 
  \end{align*}
  We have that $(a_{1}e_{1} + a_{2}e_{2} + a_{3}e_{3}) \wedge (b_{1}e_{1} + b_{2}e_{2} + b_{3}e_{3}) = (a_{2}b_{3} - a_{3}b_{2})e_{1}\wedge e_{2} + (a_{1}b_{3} - a_{3}b_{1}) e_{2} \wedge e_{3} + (a_{1}b_{2} - a_{2}b_{1})e_{3}\wedge e_{1} = \omega_{1}e_{1}\wedge e_{2} + \omega_{2} e_{2}\wedge e_{3} + \omega_{3} e_{3} \wedge e_{1}$. As given any vector in $\mathbb{R}^{3}$ we may obtain two vectors that are orthogonal to it and thus that cross product to it, we have that the above system of equations has a solution, and thus any 2-form may be expressed as an indecomposable one. 
\end{proof}

\end{document}