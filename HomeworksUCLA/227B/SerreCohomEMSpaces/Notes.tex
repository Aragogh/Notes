\documentclass{article}

\usepackage{fancyhdr}
\usepackage{extramarks}
\usepackage{amsmath}
\usepackage{amsthm}
\usepackage{amssymb}
\usepackage{amsfonts}
\usepackage{tikz}
\usepackage[plain]{algorithm}
\usepackage{algpseudocode}
\usepackage{nameref}
\usepackage{cite}
\usepackage{tikz-cd}
\usepackage{mathrsfs}
\documentclass{article}

\usepackage{fancyhdr}
\usepackage{extramarks}
\usepackage{amsmath}
\usepackage{amsthm}
\usepackage{amssymb}
\usepackage{amsfonts}
\usepackage{tikz}
\usepackage[plain]{algorithm}
\usepackage{algpseudocode}
\usepackage{nameref}
\usepackage{cite}
\usepackage{tikz-cd}
\usepackage{mathrsfs}

\usetikzlibrary{automata,positioning}


\topmargin=-0.45in
\evensidemargin=0in
\oddsidemargin=0in
\textwidth=6.5in
\textheight=9.0in
\headsep=0.25in


\renewcommand\headrulewidth{0.4pt}
\renewcommand\footrulewidth{0.4pt}
\newcommand{\sur}[1]{\ensuremath{^{\textrm{#1}}}}
\newcommand{\sous}[1]{\ensuremath{_{\textrm{#1}}}}
\newcommand{\Hom}{\text{Hom}}
\newcommand{\bb}[1]{\mathbb{#1}}

\newtheorem{lemma}{Lemma}
\newtheorem{theorem}{Theorem}
\newtheorem{exercise}{Exercise}
\newtheorem{Corollary}{Corollary}

\renewcommand\headrulewidth{0.4pt}
\renewcommand\footrulewidth{0.4pt}
\newcommand{\sur}[1]{\ensuremath{^{\textrm{#1}}}}
\newcommand{\sous}[1]{\ensuremath{_{\textrm{#1}}}}
\newcommand{\Hom}{\text{Hom}}
\newcommand{\bb}[1]{\mathbb{#1}}

\setlength\parindent{0pt}

\begin{document}
\begin{center}
  \Huge \textbf{On the Cohomology of Eilenberg-Maclane Spaces Modulo 2}
\end{center}

\vspace{0.2in}
Things I hope to address by the end of today's talk:
\begin{itemize}
\item Serre's Computation of $H^{*}(\bb{Z}/2, n ; \bb{Z}/2)$ and immediate consequences
\item If time, we'll examine applications to computing stable homotopy groups 
\end{itemize}

\underline{Transgression}

Let's run a quick refresher on the transgression:
\[
\begin{tikzcd}
 \arrow[r] & \pi_{q}(F) \arrow[dd, "h"] \arrow[r, "i"]      & \pi_{q}(E) \arrow[r, "p"] \arrow[rd] \arrow[dd, "h"]    & \pi_{q}(B) \arrow[r, "\partial"] \arrow[d]                     & \pi_{q-1}(F) \arrow[dd, "h"] \arrow[ld] \\
           &                                                &                                                         & {\pi_{q}(E,F)} \arrow[d, "h"]                           &                                         \\
 \arrow[r] & H_{q}(F) \arrow[r, "i_{*}"] \arrow[d, "p_{*}"] & H_{q}(E) \arrow[r, "j_{*}"] \arrow[d, "p_{*}"]          & {H_{q}(E,F)} \arrow[r, "\partial"] \arrow[d, "p_{0 *}"] & H_{q-1}(F) \arrow[d, "p_{*}"]           \\
 \arrow[r] & H_{q}(*) \arrow[r, "i_{*}"]                    & H_{q}(B) \arrow[r, "j_{*}"] \arrow[rru, "\tau", dotted] & {H_{q}(B,*)} \arrow[r, "\partial"]                      & H_{q-1}(*)                             
\end{tikzcd}
\]

Which is a map $\tau: i_{*}^{-1}\text{im}(p_{0}_{{*}}) \to H_{q-1}(F)/\partial(ker (p_{{0}}))$, which is well-defined by the first isomorphism. In the cohomological Serre Spectral Sequence we note that this can be realized as a map $\tau: E^{0,n-1}_{2} \to E^{n,0}_{2}$, and by noting that the first quadrant-ness means that $E_{\infty}^{0,n-1} \cong ker(d_{n})\subset E_{n}^{0,n-1}, E_{\infty}^{n,0} \cong E_{n}^{n,0}/im(d_{n})$ and chasing definitions gives that this is in fact the subobject and quotient of the bits in the $E_{2}$-page (subobject and quotient as one always has trivial boundary, one has everything cycle) that we care about, and the transgression is literally just the differential map $d_{n}$.
\\

Call $x_{i}$ a simple system of generators of a cohomology ring if they are homogenous and products of distinct elements $x_{1} \cdot ... \cdot x_{n}$ ARE A BASIS OF the cohomology ring as a vector space over $\bb{Z}/2$
We have the result of Borel's Theorem that we have used in the past for classifying space cohomology, given by:

\begin{theorem}
  $F\to E \to B$ fibration with $B$ path connected s.t. \\
  1) $E_{2}$ page gien by $H^{*}(B, \bb{Z}/2) \otimes H^{*}(B, \bb{Z}/2$ (simple system of local coefficients) \\
  2) $H^{i}(E, \bb{Z}/2) = 0$ \forall \ i > 0 \\
  3) $H^{*}(F, \mathbb{Z}/2)$ has a simple system of transgressive generators $(x_{i})$ then \\
  For x$_{i}$ transgresses to $y_{i} \in H^{*}(B, \bb{Z}/2)$ we have that the latter is polynomial generated by $y_{i}$. 
\end{theorem}

We now use our favorite fibration $K(\bb{Z}/2,n-1) \to PK \to K(\bb{Z}/2, n)$ \\

Realization: an easy check of the above sequence shows that the fundamental class of the fiber is transgressive, and transgresses to the fundamental class of the base: the $Sq^{i}$ commute with $\tau$ by naturality: so if $\tau(\iota_{n-1}) = \tau(\iota_{n})$ then $\tau(Sq^{i}\iota_{n-1}) = Sq^{i}(\tau(iota_{n-1}) = Sq^{i}\iota_{n}$ which is nonzero as long as $Sq^{i}$ is defined and not identically on grade $n$ (again, naturality shows that you cannot pull back a trivial class to a nontrivial one). \\

This is quite nice, as for any polynomial algebra $K[x_{1},x_{2}...,x_{n}]$ we have a simple system of generators given by $(x_{1}, x_{1}^{2}, x_{1}^{2^{2}},...,x_{n}, x_{n}^{2}, x_{n}^{2^{2}})$ the products of which form a basis by essentially binary decomposition. Noting that when this polynomial algebra is the cohomology ring of some fiber on transgressive generators, we have that each generator is iterating cup product squares, i.e. the top steenrod square on each thing, and as a result we have a simple system of transgressive generators. \\

Call a sequence $I = i_{1},...,i_{r}$ admissible if $i_{n} \geq 2i_{n+1}$, and let the excess $e(I) = (i_{1} - 2_{i_{2}}) + ... + (i_{r-1} - 2i_{r}) + i_{r} = 2i_{1} - \sum_{r}i_{k}$.   
\begin{theorem}
  The cohomology ring $H^{*}(\mathbb{Z}/2, q; \mathbb{Z}/2) \cong \bb{Z}/2[Sq^{I}]$ where $I$ runs through all admissible sequences s.t. $e(I) < q$.
\end{theorem}
\\

The proof of this is an induction noting that $H^{*}(\bb{Z}/2, 1) = \bb{Z}/2[\iota_{1}]$ real projective space, so the theorem is clearly true here. Supposing it holds for $q-1$, then the work upstairs gives us that the cohomology ring of the total space comprises iterated cup squares of $Sq^{J}(\iota_{q})$ for $e(J) < q-1$, and the sequences we care about are of thus of the form $\{2^{k}s_{J},...,s_{J},j_{1},...,j_{r}\}$ for $s_{J}$ the degree of that guy. A little work shows that every admissible sequence of $e(I) < q$ is represented there exactly once and we have the result. \\

We apply a slight modification to derive $H^{*}(\bb{Z},n; \bb{Z}/2)$, where the result is the same except for the caveat that the last term of the admissible sequence must be greater than 1: starting the above induction at $\bb{CP}^{\infty}$ shows the result. \\

\underline{Applications to Cohomology Operations} \\

From the above result, we derive the following:
\begin{theorem}
  All operations $C: H^{q}(X, \bb{Z}/2) \to H^{n}(X,\bb{Z}/2)$ are of the form $C(x) = p(Sq^{I_{1}},...,Sq^{I_{k}})(x)$
\end{theorem}
\begin{Corollary}
  If $n \leq 2q$ then for $I$ running througha admissible sequences of degree n the $Sq^{I}$ compose a basis for operations of type $(q, n, \bb{Z}/2, \bb{Z}/2)$.
\end{Corollary} \\

The above corollary also tells us that for $q > n$, $H^{n+q}(\mathbb{Z}/2,q;\bb{Z}/2)$ is generated by the admissible sequences of degree $n$, and so all stable operations must eventually suspend to some admissible sequence in the squares, which commmute with stability, so these are in fact the stable operations we care about. \\

This computation gives us an axiomatic description of the squares, as they can entirely be determined by their actions on  
We can also derive a proof of the Adem relations, by taking the map $f: (\bb{CP}^{\infty})^{\times n} \to K(\mathbb{Z}/2, n)$ corresponding to $f^{*}(\iota_{n}) = x_{1}\cdot...\cdot x_{n}$ for $x_{i}$ the generator of the $i$th component. A quick computation by iterating the Cartan formula, essentially, shows that the $Sq^{I}(x_{1}\cdot ... \cdot x_{n}$ are symmetric polynomials and end up being linearly independent over the symmetric polynomials and thus the polynomial ring $\bb{Z}/2[x_{1},...,x_{n}]$; it is thus possible to derive relations between them in this fashion.
\end{document}
