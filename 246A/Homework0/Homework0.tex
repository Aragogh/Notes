\documentclass{article}

\usepackage{fancyhdr}
\usepackage{extramarks}
\usepackage{amsmath}
\usepackage{amsthm}
\usepackage{amssymb}
\usepackage{amsfonts}
\usepackage{tikz}
\usepackage[plain]{algorithm}
\usepackage{algpseudocode}
\usepackage{nameref}
\usepackage{cite}
\usepackage{tikz-cd}
\usepackage{mathrsfs}
\usepackage{tikz}
\newcommand*\circled[1]{\tikz[baseline=(char.base)]{
            \node[shape=circle,draw,inner sep=2pt] (char) {#1};}}

\usetikzlibrary{automata,positioning}


\topmargin=-0.45in
\evensidemargin=0in
\oddsidemargin=0in
\textwidth=6.5in
\textheight=9.0in
\headsep=0.25in

\linespread{1.1}

\pagestyle{fancy}
\chead{\hmwkTitle}
\lhead{\hmwkAuthorName}
\rhead{\hmwkClass}
\cfoot{\thepage}

\renewcommand\headrulewidth{0.4pt}
\renewcommand\footrulewidth{0.4pt}
\newcommand{\sur}[1]{\ensuremath{^{\textrm{#1}}}}
\newcommand{\sous}[1]{\ensuremath{_{\textrm{#1}}}}
\newcommand{\Hom}{\text{Hom}}
\newcommand{\Tor}{\text{Tor}}
\newcommand{\Ext}{\text{Ext}}
\newcommand{\bb}[1]{\mathbb{#1}}
\newcommand{\fk}[1]{\mathfrak{#1}}
\newcommand{\iso}{\cong}

\setlength\parindent{0pt}

%c
% Create Problem Sections
%

\newtheorem{lemma}{Lemma}
\newtheorem{exercise}{Exercise}
%
% Homework Details
%   - Title
%   - Due date
%   - Class
%   - Section/Time
%   - Instructor
%   - Author
%

\newcommand{\hmwkTitle}{Homework 0}
\newcommand{\hmwkDueDate}{Oct 4th, 2019}
\newcommand{\hmwkClass}{Math 246A Complex Analysis}
\newcommand{\hmwkClassInstructor}{Professor Rowan Killip}
\newcommand{\hmwkAuthorName}{\textbf{Anish Chedalavada}}

%
% Title Page
%

\title{
    \vspace{2in}
    \textmd{\textbf{\hmwkClass:\ \hmwkTitle}}\\
    \vspace{0.1in}
    \textmd{\hmwkDueDate} \\
    \vspace{0.2in}\large{\textit{\hmwkClassInstructor\  }}
    \vspace{2in}
}

\author{\hmwkAuthorName}
\date{}

\begin{document}
\maketitle
\newpage
\begin{exercise}
  a) Fix $\lambda \in \bb{R}$ and $a,b \in \bb{C}$ with $\lambda > 0, \ \lambda \neq 1, \ a \neq b$. Use algebraic manipulations to identify \[ \left\{ z \in \bb{C} : \left| \frac{z-a}{z-b} \right| = \lambda\right\} \] as a circle.\\
  b) Show that every circle can be realized in this manner. \\
  c) Give analogues of a) and b) when $\lambda = 1$.
\end{exercise}
\begin{proof}
 a) Consider $\frac{z-a}{z-b}$ in the set given above. We have that $|\frac{z-a}{z-b}| = \lambda \implies |z-a| = \lambda |z-b|$. Set $z = x + iy, \ a = a_{1} + ia_{2}, \ b = b_{1} + ib_{2}$. We have from the formula above the following derivation:
  \begin{align*}
    & |z-a| = \lambda |z-b | \implies (x-a_{1})^{2} + (x-a_{2})^{2} = \lambda^{2}(x-b_{1})^{2} + \lambda^{2}(x-b_{2})^{2} \\
    \rightarrow \ & (x-a_{1})^{2} - \lambda^{2}(x-b_{1})^{2} - x^{2}(1-\lambda^{2}) - 2x(a_{1} - \lambda^{2}b_{1}) + a_{1}^{2} - \lambda^{2}b_{1} = 0 \text{ \ --- \ } \circled{1} \\
    \rightarrow \ &  x^{2} - \frac{2x(a_{1} - \lambda^{2}b_{1})}{(1-\lambda^{2})} +  \frac{(a_{1}^{2} - \lambda^{2}b_{1})}{(1-\lambda^{2})} \\
    = \ & x^{2} - \frac{2x(a_{1} - \lambda^{2}b_{1})}{(1-\lambda^{2})} +  \frac{a_{1}^{2} - \lambda^{2}(a_{1} + b_{1}) +\lambda^{4}b_{1}^{2} - 2\lambda^{2}a_{1}b_{1} + 2\lambda^{2}a_{1}b_{1}}{(1-\lambda^{2})^{2}} \\
    = \ & x^{2} - \frac{2x(a_{1} - \lambda^{2}b_{1})}{(1-\lambda^{2})} +  \frac{(a_{1}-\lambda^{2}b_{1})^{2}}{(1-\lambda^{2})^{2}} - \frac{\lambda^{2}(a_{1} + b_{1})^{2}}{(1-\lambda^{2})^{2}} = \left(x - \frac{a_{1} + \lambda^{2}b_{1}}{1-\lambda^{2}}\right) ^{2} - \ \left(\frac{\lambda(a_{1} + b_{1})}{1-\lambda^{2}}\right)^{2}
\end{align*}
Similarly simplifying for the parts of the equation involving $y$, we have by rewriting \circled{1} that :
\begin{align*}
  \rightarrow \ &\left(x + \frac{a_{1} - \lambda^{2}b_{1}}{1-\lambda^{2}}\right) ^{2} - \ \left(\frac{\lambda(a_{1} + b_{1})}{1-\lambda^{2}}\right)^{2} + \ \left(y - \frac{a_{2} + \lambda^{2}b_{2}}{1-\lambda^{2}}\right) ^{2} - \  \left(\frac{\lambda(a_{2} + b_{2})}{1-\lambda^{2}}\right)^{2} = 0 \\
  \implies \ & \left(x - \frac{a_{1} + \lambda^{2}b_{1}}{1-\lambda^{2}}\right) ^{2} + \ \left(y - \frac{a_{2} + \lambda^{2}b_{2}}{1-\lambda^{2}}\right) ^{2} =  \left(\frac{\lambda(a_{1} + b_{1})}{1-\lambda^{2}}\right)^{2} + \ \left(\frac{\lambda(a_{2} + b_{2})}{1-\lambda^{2}}\right)^{2}  
\end{align*}
Which is of the form a circle centered at the point \[\left(\frac{a_{1}+\lambda^{2}b_{1}}{1-\lambda^{2}}, \frac{a_{1}+\lambda^{2}b_{1}}{1-\lambda^{2}}\right)\] with radius \[ \left(\frac{\lambda(a_{1} + b_{1})}{1-\lambda^{2}}\right)^{2} + \ \left(\frac{\lambda(a_{2} + b_{2})}{1-\lambda^{2}}\right)^{2} \] 
\\ 
b) Suppose we are given a circle of radius $r$ centered at a point $(a, b)$. 
\end{proof}

\begin{exercise}
  Show algebraically for every triple $a,b,c$ of distinct unimodular complex numbers, \[\frac{b-a}{1-\bar{a}b} = \frac{c-a}{1-\bar{a}c} \] Show that with a little further manipulation this expresses the inscribed angle theorem.
\end{exercise}
\begin{proof}
  We have that for $x=a,b,c$, $|x|^{2} = 1$. Consider the following string of manipulations:
  \[
    \frac{1-\bar{a}b}{1-\bar{a}c} = \frac{1-\bar{a}b}{1-\bar{a}c} \cdot \frac{a}{a} = \frac{a-b}{a-c} = \frac{b-a}{c-a} \implies \frac{b-a}{1-\bar{a}b} = \frac{c-a}{1-\bar{a}c} 
  \]
  Now for the second claim, consider the following diagram:
  
\tikzset{every picture/.style={line width=0.75pt}} %set default line width to 0.75pt        

\begin{tikzpicture}[x=0.75pt,y=0.75pt,yscale=-1,xscale=1]
%uncomment if require: \path (0,444); %set diagram left start at 0, and has height of 444

%Shape: Axis 2D [id:dp3251959569086216] 
\draw  (28.5,218.41) -- (647.5,218.41)(336.96,14) -- (336.96,414) (640.5,213.41) -- (647.5,218.41) -- (640.5,223.41) (331.96,21) -- (336.96,14) -- (341.96,21)  ;
%Shape: Ellipse [id:dp4816851887951721] 
\draw   (238.23,218.41) .. controls (238.23,164.09) and (282.43,120.05) .. (336.96,120.05) .. controls (391.49,120.05) and (435.69,164.09) .. (435.69,218.41) .. controls (435.69,272.72) and (391.49,316.76) .. (336.96,316.76) .. controls (282.43,316.76) and (238.23,272.72) .. (238.23,218.41) -- cycle ;
%Straight Lines [id:da09826691019476108] 
\draw    (401.06,145.41) -- (336.96,218.41) ;


%Straight Lines [id:da8002185927378136] 
\draw    (401.06,145.41) -- (244.5,259.49) ;


%Straight Lines [id:da9748457475539547] 
\draw    (435.69,218.41) -- (244.5,259.49) ;


%Shape: Arc [id:dp5891500147127133] 
\draw  [draw opacity=0] (352.45,199.64) .. controls (359.03,203.05) and (363.68,209.98) .. (364.35,218.09) -- (343.05,220.2) -- cycle ; \draw   (352.45,199.64) .. controls (359.03,203.05) and (363.68,209.98) .. (364.35,218.09) ;
%Shape: Arc [id:dp09073129949045988] 
\draw  [draw opacity=0] (272.24,238.21) .. controls (275.09,241.73) and (276.95,246.19) .. (277.35,251.09) -- (256.05,253.2) -- cycle ; \draw   (272.24,238.21) .. controls (275.09,241.73) and (276.95,246.19) .. (277.35,251.09) ;

% Text Node
\draw (238.5,263.49) node  [align=left] {a};
% Text Node
\draw (411,134) node  [align=left] {c};
% Text Node
\draw (445,228) node  [align=left] {b};
% Text Node
\draw (371,198) node  [align=left] {$\displaystyle \phi $};
% Text Node
\draw (286,262) node  [align=left] {$\displaystyle \psi $};

\end{tikzpicture}
\\

Our objective is to show that $2\psi = \phi$. Note that $\psi$ is the argument of the complex number given by $\frac{c-a}{1-a}$ ($b$ = 1). Dividing this complex number by its conjugate will yield $e^{i2\psi}$, i.e. a unimodular complex number with twice the argument. We have that following set of manipulations:
\[ e^{i2\psi} = \frac{c-a}{1-a}\cdot \frac{1-\bar{a}}{\bar{c}-\bar{a}} \cdot \frac{c}{c} = \frac{c-a}{1-\bar{a}c} \cdot \frac{c(1-\bar{a})}{1-a} = \frac{1-a}{1-\bar{a}} \cdot \frac{c(1-\bar{a})}{1-a} = c = e^{i\phi} 
  \]
And thus $2 \psi = \phi$.
\end{proof}

\begin{exercise}
  Give a proof of the Intersecting Cord Theorem of Jakob Steiner.
\end{exercise}
\begin{proof}
  
\end{proof}

\begin{exercise}
  Any mapping that can be represented in the form \[z \mapsto \frac{az + b}{cz + d}\] with $ad -bc \neq 0$ is called a Mobius transformation. \\
a) Show that every such mapping can be realised by coefficients satisfying $ad-bc = 1$ and determine the number of such representations.
\end{exercise}
\begin{proof}
  a) Given a transformation of the above form, note that $ad-bc = k \neq 0$. As the complex numbers are algebraically closed, we may select $w$ s.t. $w^{2} = \frac{1}{k}$. Multiplying both the numerator and the denominator by $w$ yields the same transformation with coefficients $a', b', c', d'$ s.t. $a'd' - b'c' = \frac{1}{k^{2}}(ad-bc) = 1$. Now suppose there exists another set of coefficients $a'',b'', c'', d''$ representing the transformation s.t. $a''d'' - b''c'' = 1$. Note that for $z = \frac{-b'}{a'}$, we have that both transformations must send $z$ to 0, which is only possible if the numerator is sent to $0$ and thus $a''z + b'' = 0 \implies z = \frac{-b''}{a''}$ and similarly for $z = \frac{-d'}{c'}$ yielding a pole at $z$.    
  \end{proof}
\end{document}