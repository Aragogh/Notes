\documentclass{article}

\usepackage{fancyhdr}
\usepackage{extramarks}
\usepackage{amsmath}
\usepackage{amsthm}
\usepackage{amssymb}
\usepackage{amsfonts}
\usepackage{tikz}
\usepackage[plain]{algorithm}
\usepackage{algpseudocode}
\usepackage{nameref}
\usepackage{cite}
\usepackage{tikz-cd}
\usepackage{mathrsfs}

\usetikzlibrary{automata,positioning}


\topmargin=-0.45in
\evensidemargin=0in
\oddsidemargin=0in
\textwidth=6.5in
\textheight=9.0in
\headsep=0.25in

\linespread{1.1}

\pagestyle{fancy}
\chead{\hmwkTitle}
\lhead{\hmwkAuthorName}
\rhead{\hmwkClass}
\cfoot{\thepage}

\renewcommand\headrulewidth{0.4pt}
\renewcommand\footrulewidth{0.4pt}
\newcommand{\sur}[1]{\ensuremath{^{\textrm{#1}}}}
\newcommand{\sous}[1]{\ensuremath{_{\textrm{#1}}}}
\newcommand{\Hom}{\text{Hom}}
\newcommand{\Tor}{\text{Tor}}
\newcommand{\Ext}{\text{Ext}}
\newcommand{\bb}[1]{\mathbb{#1}}
\newcommand{\fk}[1]{\mathfrak{#1}}
\newcommand{\iso}{\cong}

\setlength\parindent{0pt}

%c
% Create Problem Sections
%

\newtheorem{lemma}{Lemma}
\newtheorem{exercise}{Exercise}
%
% Homework Details
%   - Title
%   - Due date
%   - Class
%   - Section/Time
%   - Instructor
%   - Author
%

\newcommand{\hmwkTitle}{Homework 2}
\newcommand{\hmwkDueDate}{Sep 9th, 2020}
\newcommand{\hmwkClass}{Math 552}
\newcommand{\hmwkClassInstructor}{Professor Kevin Tucker}
\newcommand{\hmwkAuthorName}{\textbf{Anish Chedalavada}}

%
% Title Page
%

\title{
    \vspace{2in}
    \textmd{\textbf{\hmwkClass:\ \hmwkTitle}}\\
    \vspace{0.1in}
    \textmd{\hmwkDueDate} \\
    \vspace{0.2in}\large{\textit{\hmwkClassInstructor\  }}
    \vspace{2in}
}

\author{\hmwkAuthorName}
\date{}

\begin{document}

\maketitle

\newpage
\begin{exercise}
  Prove the following: \\
  a) (Cayley-Hamilton Trick) Suppose that $M$ is a finitely generated module over a commutative ring $R$. If $\phi$ is any endomorphism, then $\phi$ satisfies a monic relation. \\
  b) (Zariski's Lemma) if $R \subseteq S$ is a finite extension of domains, then $R$ is a field iff $S$ is a field.
\end{exercise}
\begin{proof}
  a) Suppose $M$ is finitely generated over $R$. In particular, we have an epimorphism from the free module $\pi: R^{m} \twoheadrightarrow M$ given by selection of the generators of $M$. Given an endomorphism $\phi: M \to M$, consider the following diagram:\\
\[  
\begin{tikzcd}
                                                                        &                     & R^m \arrow[d, "\pi", two heads] \arrow[r, "\widetilde{\phi}", dotted] & ... \arrow[r, "\widetilde{\phi}", dotted] & R^m \arrow[d, "\pi", two heads] \\
R^m \arrow[r, "\pi", two heads] \arrow[rru, "\widetilde{\phi}", dotted] & M \arrow[r, "\phi"] & M \arrow[r, "\phi"]                                                   & ... \arrow[r, "\phi"]                     & M                              
\end{tikzcd}
\]

We have the lift $\widetilde{\phi}$ via the universal property of projective modules ($\pi$ is surjective), and as this is a free module homomorphism we may represent it by an $m \times m$ matrix $A$. By the Cayley-Hamilton theorem, we have that $p(x) = det(xI - A)$ for $I$ the identity matrix is a vanishing polynomial for $\widetilde{\phi}$. In particular, this is a monic polynomial in $R$. Note that by the diagram above, $\pi \circ \widetilde\phi = \phi \circ \pi$, and iterating $\widetilde\phi$ as in the diagram above yields $\pi \circ \widetilde\phi^{n} = \phi^{n} \circ \pi$. Thus, we have that $p(\phi) \circ \pi = \pi \circ p(\widetilde\phi) = 0$ by the logic above, and as $\pi$ is a surjective we have that $p(\phi) \circ \pi = 0 \implies p(\phi) = 0$, yielding the claim. \\

b) Assume $R$ is a field. Given an arbitrary nonzero element $s \in S$, we have that $s$ satisfies some minimal monic polynomial $p$ over $R$ by finiteness, and thus we have a morphism $R[x]/(p) \hookrightarrow S$ given by $x \mapsto s$ (inclusion as $p$ assumed minimal, i.e. that $(p)$ is the kernel). However, as $S$ is assumed to be a domain we have that $(p)$ must be a prime ideal and thus maximal as $R[x]$ is a PID ($R$ assumed to be a field). Thus, $R[s] \subset S$ is a field, implying that $s$ has a multiplicative inverse in $S$. As $s \in S$ was arbitrary nonzero, this is true of all $s \in S \setminus \{0\}$ and thus $S$ is a field. \\
Assume now that $S$ is a field. Suppose $R$ is not a field, i.e. that there exists some proper maximal ideal $(0) \neq \fk{m} \subset R$. Then for any $0 \neq m \in \fk{m}$, we have that $m^{-1} \in S$ satisfies a monic relation in $R$ (by finiteness) given by:
\[
  m^{-n} + a_{n-1}m^{-(n-1)} + ....+ a_{0} = 0
\]
Multiplying the entire above equation by $m^{n}$ yields:
\[
  1 + a_{n-1}m + ... + a_{0}m^{n} = 0 \implies 1 = a_{n-1}m + ... + a_{0}m^{n} \implies 1 \in \fk{m}
\]

Yielding a contradiction, as $\fk{m}$ assumed to be a proper maximal ideal. Thus, we cannot select $0 \neq m \in \fk{m}$, implying that $\fk{m}$ must be zero and thus the only maximal ideal in $R$ is $(0)$, implying that $R$ is a field.
\end{proof}

\begin{exercise}
  Let $k = \overline{k}$ be a field of characteristic not 2. Let $V_{c}$ be the intersection of the circle $x^{2}+y^{2} = 1$ in $\bb{A}^{2}$ with the line $x=c$, for $c \in k$. \\
  a) Find $I(V_{c})$. \\
  b) Consider the projection $\bb{A}^{2} \to \bb{A}^{1}$ sending $(x,y) \mapsto x$, restricted to circle $x^{2} + y^{2} =1$. Draw a picture to describe this map, and interpret the variety $V_{c}$ geometrically in terms of this map. \\
  c) What happens in characteristic 2?
\end{exercise}
\begin{proof}
  a) In a field of characteristic $2$, we have that $\sqrt{(x^{2}+y^{2}-1)} = (x^{2}+y^{2}-1)$, as there does not exist $p \in k[x,y]$ such that $p^{n} | x^{2}+y^{2}-1$ for $n \neq 1$ (required in a UFD). Thus, $I(x^{2}+y^{2} = 1) = (x^{2}+y^{2}-1)$ by the Nullstellensatz. Thus, we have that $I(\{x^{2}+y^{2} = 1\} \cap \{x=c\}) = (x^{2}+y^{2} -1, x-c) = (y^{2}+(c^{2}-1), x-c)$. For $c \neq \pm 1$, we have that $y^{2} = c^{2}-1$ has two solutions $a_{1}, a_{2}$ for $y$ in $k[y]$, and we have two irreducible components of $\{x^{2}+y^{2} = 1\} \cap \{x=c\}$ corresponding to the points $(c, a_{1}), \ (c,a_{2})$ for the prime ideals $(y-a_{1}, x-c), \ (y-a_{2}, x-c)$ which contain and intersect in our ideal. For $c = \pm 1$ we have that $c^{2} - 1 = 0$ and $y^{2} = c^{2} - 1$ has only one solution with multiplicity at $0$; i.e. that $\{x^{2}+y^{2} = 1\} \cap \{x=c\} = (\pm 1, 0)$ corresponding to $\sqrt{(y^{2}, x-c)} = (y, x-c)$. \\
  
  b) Consider the following diagram:

    \begin{center}

\tikzset{every picture/.style={line width=0.75pt}} %set default line width to 0.75pt        

\begin{tikzpicture}[x=0.75pt,y=0.75pt,yscale=-1,xscale=1]
%uncomment if require: \path (0,300); %set diagram left start at 0, and has height of 300

%Shape: Axis 2D [id:dp22133796006929618] 
\draw  (50,163) -- (575.5,163)(311.5,12) -- (311.5,295) (568.5,158) -- (575.5,163) -- (568.5,168) (306.5,19) -- (311.5,12) -- (316.5,19)  ;
%Shape: Circle [id:dp49991287310435073] 
\draw   (254,163) .. controls (254,131.24) and (279.74,105.5) .. (311.5,105.5) .. controls (343.26,105.5) and (369,131.24) .. (369,163) .. controls (369,194.76) and (343.26,220.5) .. (311.5,220.5) .. controls (279.74,220.5) and (254,194.76) .. (254,163) -- cycle ;
%Curve Lines [id:da2511415979135525] 
\draw    (70.5,76) .. controls (178.5,76) and (257.5,112) .. (254,163) ;
%Curve Lines [id:da2902835516520641] 
\draw    (57.5,240) .. controls (165.5,240) and (253.5,220) .. (254,163) ;
%Curve Lines [id:da5335746431920938] 
\draw    (571.5,65) .. controls (468.5,73) and (366.5,106) .. (369,163) ;
%Curve Lines [id:da15086780431121083] 
\draw    (566.5,248) .. controls (496.5,241) and (368.5,220) .. (369,163) ;
%Straight Lines [id:da9366872715796133] 
\draw    (369.5,10) -- (368.7,239) -- (368.5,295) ;
%Straight Lines [id:da14992796044749146] 
\draw    (254.5,8) -- (253.7,237) -- (253.5,293) ;

% Text Node
\draw (187,15.4) node [anchor=north west][inner sep=0.75pt]    {$x\ =\ -1$};
% Text Node
\draw (378,14.4) node [anchor=north west][inner sep=0.75pt]    {$x\ =\ 1$};


\end{tikzpicture}

\end{center}

Above is the graph of our algebraic set $x^{2}+y^{2} = 1$, and at $x =1, -1$ we have that there is only one solution with a double point singularity, i.e. only one point in the fiber over $x = 0, y = 0$ under the projection. \\

c) If the characteristic of $k$ is 2, then the ideal $I(V_{c})$ contains $(x^{2}+y^{2} + 1, x + c) = (y^{2} + c^{2}+1, x+c) = ((y+c+1)^{2}, x + c)$, the radical of which is $(y+c+1, x + c)$, corresponding to the single irreducible component $(c, c + 1)$.  Thus, each point in the $x$-axis has only one point in its fiber under the projection map from the curve.
\end{proof}
\begin{exercise}
Consider the map $\bb{A}^{1} \to \bb{A}^{3}$ given by $t \mapsto (t,t^{2},t^{3})$. Show that the image of this map is an affine algebraic set that can be defined by two equations in $k[x,y,z]$.  
\end{exercise}
\begin{proof}
  Consider the equations $x^{2} - y, \ x^{3} - z \in k[x,y,z]$. We have that the vanishing set of these equations is all points $(a,b,c)$ such that $b = a^{2}, \  c = a^{3}$; i.e. the points $(a,a^{2},a^{3}) \in \bb{A}^{3}$ for $a \in k$. Thus, we have that the image of the map in the question is exactly an algebraic set, the vanishing of which is given by the equations above. 
\end{proof}

\begin{exercise}
  Describe the image of the map $f: \bb{A}^{2} \to \bb{A}^{2}$ given by $f: (x,y) \mapsto (x, xy)$. Is it open, closed, and/or dense?
\end{exercise}
\begin{proof}
  The image of $f$ is given by $\bb{A}^{2} \setminus \{(0,y) \ | \ y \neq 0\}$, as for any $(x,y) \in \bb{A}^{2}$ s.t. $x$ is a unit, w.h.t. $f(x,y/x) = (x,y)$. For $x = 0, w.h.t. f(x,y) = (0,0)$. Consider the map $g_{z}: \bb{A}^{1} \to \bb{A}^{2}$ via $g: t \mapsto (t,n)$ (given by the map on coordinate rings given by $k[x,y] \to k[x]$ by $y \mapsto n \in k$); we have that $g_{3}^{-1}(Im(f)) = \bb{A}^{2} \setminus \{0\}$ which is open in the Zariski topology, being the complement of the set $\{t = 0\}$, thus  $Im(f)$ is not a closed set. We also have the map $h:\bb{A}^{1} \to \bb{A}^{2}$ by $h: t \mapsto (0,t)$, and $h^{-1}(Im(f)) = \{0\}$, which is closed in the Zariski topology; thus, $Im(f)$ is neither open nor closed in the Zariski topology. Finally, we have that for $g: X \to Y$ a continuous map and $A \subset X$, $\overline{g^{-1}(A)} \subseteq g^{-1}(\overline{A})$. Thus for any $n \neq 0$ in $k$, w.h.t. $\overline{g_{n}^{-1}(A)} \subseteq g_{n}^{-1}(\overline{Im(f)})$, and for $n = 0$ w.h.t. $g^{-1}(Im(f)) = \bb{A}^{2} \setminus \{0\}$, and thus $\overline{g^{-1}(\bb{A}^{2})} = \bb{A}^{2} \subset g^{-1}(\overline{Im(f)})$, implying $\forall \ t \in k, \ (0,t) \in \overline{Im(f)}$, and thus $Im(f) = \bb{A}^{2}$, so the set is dense. 
\end{proof}
\newpage
\begin{exercise}
  For the morphisms in 2 b), 3, and 4, describe the induced map of coordinate rings.
\end{exercise}
\begin{proof}
  The morphism in 2 b) is the projection of the curve $C = x^{2}+y^{2}=1$ onto the $x$-axis. We have that the coordinate ring of this curve is given by $k[x,y]/(x^{2}+y^{2}=1)$. We have that the projection factors as $C \hookrightarrow \bb{A}^{2} \twoheadrightarrow \bb{A}^{1}$, inducing a map on coordinate rings by $k[x] \hookrightarrow k[x,y] \twoheadrightarrow k[x,y]?(x^{2}+y^{2} = 1)$. Thus, the map of coordinate rings is given by $k[x] \to k[x,y]/(x^{2}+y^{2} = 1)$ via $x \mapsto [x]$. \\

  For the morphism in 3, we have that the coordinate ring of the image of $\bb{A}^{1}$ is given by $k[x,y,z]/(x^{2}-y, x^{3}-z) \iso k[x]$. We have that the functions $y,z$ pull back to the functions $x^{2}, x^{3}$ via this map, and so the map on coordinate rings is explicitly given by the quotient map $k[x,y,z] \to k[x,y,z]/(x - y^{2}, x - z^{3}) \iso k[x] = k[\bb{A}^{1}]$.\\

  For the morphism in 4, we have that the function $x$ pulls back along the first coordinate to the function $x$, and the function $y$ pulls back along $(x,xy)$ to be the function $xy$. This is clear as $y(a,b) = b$, and so for $f$ the map of $4$, $y(f(x,y)) = y(x,xy) = xy$. Thus, the induced map on coordinate rings is given by $k[x,y] \to k[x,y]$ via $x \mapsto x, \ y\mapsto xy$.
\end{proof}
\begin{exercise}
  Assume $k = \overline{k}$. \\
  a) What is the coordinate ring of a point? What is the $k$-algebra homomorphism corresponding to the inclusion of a point into another affine set? \\
  b) When is the coordinate ring of an affine algebraic set a finite-dimensional $k$-vector space, and what is the geometric interpretation of the same?
\end{exercise}
\begin{proof}
  a) Given a point $(a_{1},...,a_{n}) \in \bb{A}^{n}$, we have that $(x_{1}-a_{1},...,x_{n}-a_{n}) \subset I((a_{1},...,a_{n})) \subset k[x_{1},...,x_{n}]$, and as $k$ is algebraically closed we have by the Nullstellensatz that $(x_{1}-a,...,x_{n}-a_{n})$ is maximal: thus, $k[x_{1},...,x_{n}]/I((x_{1}-a_{1},...,x_{n}-a_{n})) \iso k$ is the coordinate ring of a point. The $k$-algebra homomorphism corresponding to the inclusion of the point into $\bb{A}^{n}$ is given by the quotient homomorphism $k[x_{1},...,x_{n}] \to k$ via the quotient map. \\

  b) If the coordinate ring $R$ of an affine algebraic set $X$ is a finite dimensional $k$-vector space then $R$ is Artinian. In particular, $R$ has finitely many prime ideals $\fk{m}_{1},...,\fk{m}_{n}$ and all the primes are maximal. Thus, $\sqrt{I(X)} \subset k[x_{1},...,x_{n}]$ is the intersection of finitely many maximal ideals $\fk{m}_{i}$ (which are the primes containing $I(X)$), and by the Nullstellensatz we have that $X = V(\sqrt{I(X)}) = V(\bigcap_{i=1}^{n} \fk{m}_{i}) = \bigcup_{i=1}^{n}V(\fk{m}_{i})$. By the Nullstellensatz again we have that all maximal ideals are of the form $(x_{1}-a_{1},...,x_{n}-a_{n})$ corresponding to points $(a_{1},...,a_{n}) \in \bb{A}^{n}$. Thus, $X$ is the union of finitely many points in $\bb{A}^{n}$. Furthermore, $\sqrt{I(X)} = I(X)$, \emph{again} by the Nullstellensatz, and so $I(X) = \bigcap_{i=1}^{n}\fk{m}_{i}$ and so $k[x_{1},...,x_{n}]/\bigcap_{i=1}^{n}\fk{m}_{i} = \prod_{i=1}^{n}k[x_{1},...,x_{n}]/\fk{m}_{i} = \prod_{i=1}^{n}k$; thus, the dimension of the coordinate ring is the same as the number of maximal ideals, which is the same as the number of points in $X$. 
\end{proof}

\begin{exercise}
  Show that any nonempty open subset of an irreducible topological space is dense and irreducible, and that the closure of an subset irreducible in the subspace topology is also irreducible.
\end{exercise}
\begin{proof}
  Given any two distinct proper open sets $U_{1}, U_{2} \subset X$ for $X$ an irreducible topological space, we have that $X \setminus (U_{1} \cup U_{2}) = (X \setminus U_{1}) \cup (X \setminus U_{2}) \neq X$ by irreducibility ($X$ not the union of two closed subsets), we thus have that $U_{1} \cup U_{2} \neq \emptyset$ and so any open set $U_{1}$ is dense in $X$. Now suppose there exist two distinct closed sets $V_{1}, V_{2} \subset U_{1}$ in the subspace topology s.t. $V_{1} \cup V_{2} = U_{1}$. Then there exist $\widetilde{V_{1}}, \widetilde{V_{2}} \subset X$ closed subsets such that $\widetilde{V}_{i}\cup U_{1} = V_{1}$, and $\widetilde{V_{1}} \cup \widetilde{V_{2}} \cup (X \setminus U_{1}) = X$, all of which are distinct, which would contradict the irreducibility of $X$. Thus, $U_{1}$ could not have been reducible. \\

  Now suppose $Y \subset X$ is irreducible in the subspace topology. Suppose $\overline{Y}$ is reducible, i.e. that there exist closed sets $W_{1}, W_{2} \subset X$ distinct such that $(W_{1} \cup W_{2}) \cap \overline{Y} = \overline{Y}$, then either $Y \subset W_{1}$ or $Y \subset W_{2}$ by the irreducibility of $Y$, implying that either $\overline{Y} \subset W_{1}$ or $\overline{Y} \subset W_{2}$ and thus $\overline{Y}$ is irreducible (cannot be the union of two proper closed subsets in the subspace topology.
\end{proof}
\begin{exercise}
  Let $k$ be an arbitrary field. \\
  a) If $V$ and $W$ are affine algebraic sets over $k$, show that $k[V \times W] =  k[V] \otimes_{k} k[W]$.\\
  b) Suppose that $V$ and $W$ are disjoint algebraic sets in $\bb{A}^{n}$. Show that $k[V \cup W] = k[V] \times k[W]$.
\end{exercise}
\begin{proof}
  a) Given two affine algebraic sets $V \subset \bb{A}^{n}$, $W \subset \bb{A}^{m}$, in order to know $k[V \times W] \subset k[x_{1},...,x_{n},y_{1},...,y_{m}]$ (which we'll abbreviate $k[\bb{A}^{n}\times \bb{A}^{m}]$) it suffices to understand the structure as a $k[x_{1},...,x_{m},y_{1},...,y_{m}]$-module, which is the same as a $k[\bb{A}^{n}]\otimes_{k} k[\bb{A}^{m}]$ module. Now note that we can represent $V \times W \subset \bb{A}^{n} \times \bb{A}^{m}$ as the intersection $V \times \bb{A}^{n} \cap \bb{A}^{m} \times W$. Thus, $I(V \times W)$ is given by ideal generated by $(p_{1}\otimes 1,...,p_{n}\otimes 1, 1 \otimes q_{1},...,1\otimes q_{m}) \subset k[\bb{A}^{n} \times \bb{A}^{m}]$ for $p_{1},...,p_{n}$ generating $I(V) \subset k[\bb{A}^{n}]$ and $q_{1},...,q_{m}$ generating $I(V) \subset k[\bb{A}^{m}]$. We have that that $(p_{1}\otimes 1,...,p_{n} \otimes 1)$ as a $k[\bb{A}^{n}\times \bb{A}^{m}]$ module is given by $I(V) \otimes_{k[\bb{A}^{n}]} k[\bb{A}^{n}] \otimes_{k} k[\bb{A}^{m}]$. We have that $k[\bb{A}^{n}]\otimes_{k} k[\bb{A}^{m}]$ is a free left $k[A^{n}]$ module and is thus flat, so from the following $k[\bb{A}^{n}]$ resolution,
  \[
    \begin{tikzcd}
0 \arrow[r] & I(V) \arrow[r] & {k[\bb{A}^{n}]} \arrow[r] & {k[V]} \arrow[r, "\pi_{V}"] & 0
\end{tikzcd}
    \]

  using the functor $- \otimes_{k[A^{n}]} k[\bb{A}^{n}] \otimes_{k} k[\bb{A}^{m}]: k[\bb{A}^{n}]-Mod \to k[\bb{A}^{n}]\otimes_{k}k[\bb{A}^{m}] - Mod$, we have the following diagram:
  \[
    \begin{tikzcd}
{I(V) \otimes_{k[\bb{A}^{n}]} k[\bb{A}^{n}] \otimes_{k} k[\bb{A}^{m}]} \arrow[r, hook] &  {k[\bb{A}^{n}] \otimes_{k} k[\bb{A}^{m}]} \arrow[rr, two heads, "\pi_{V} \otimes id"] & &  {k[V] \otimes_{k} k[\bb{A}^{m}]}
\end{tikzcd}
\]

i.e. that $k[\bb{A}^{n}\times \bb{A}^{m}]/(p_{1}\otimes 1,...,p_{n} \otimes 1) \iso k[V] \otimes_{k}k[\bb{A}^{m}]$. Now we have that $k[V \times W] = k[V] \otimes_{k} k[\bb{A}^{m}]/([1 \otimes q_{1}],...,[1\otimes q_{m}])$ for $[1 \otimes q_{i}] = (\pi_{V}\otimes id)(1 \otimes q_{i})$. We have that $([1 \otimes q_{1}],...,[1\otimes q_{m}]) \iso k[V]\otimes_{k}k[\bb{A}^{m}] \otimes_{k[\bb{A}^{m}]}(q_{1},...,q_{m})$, again by the same rationale as above: $k[\bb{A}^{n}] \otimes_{k} k[\bb{A}^{m}] \otimes_{k[\bb{A}^{m}]}(q_{1},...,q_{m})$ is a free left $k[\bb{A}^{n}]$ module indexed over a $k$-basis for $(q_{1},...,q_{m})$, and we have the following diagram of $k[\bb{A}^{n}] \otimes k[\bb{A}^{m}]$ modules:
\[
\begin{tikzcd}
{I(V) \otimes_{k[\bb{A}^{n}]} k[\bb{A}^{n}] \otimes_{k} k[\bb{A}^{m}] \otimes_{k[\bb{A}^{m}]}(q_{1},...,q_{m})} \arrow[d, hook] \arrow[rr, "id \otimes \iota", hook] &  & {I(V) \otimes_{k[\bb{A}^{n}]} k[\bb{A}^{n}] \otimes_{k} k[\bb{A}^{m}]} \arrow[d, hook]                                     \\
k[\bb{A}^{n}] \otimes_{k} k[\bb{A}^{m}] \otimes_{k[\bb{A}^{m}]}(q_{1},...,q_{m}) \arrow[d, "\pi_{V} \otimes id", two heads] \arrow[rr, "id \otimes \iota", hook]   &  & k[\bb{A}^{n}] \otimes_{k} k[\bb{A}^{m}] \arrow[d, "\pi_{V} \otimes id", two heads] \\
{k[V] \otimes_{k} k[\bb{A}^{m}] \otimes_{k[\bb{A}^{m}]}(q_{1},...,q_{m})} \arrow[rr, "id \otimes \iota"]                                                       &  & {k[V] \otimes_{k} k[\bb{A}^{m}]}                                                                                          
\end{tikzcd}
\]

Where $\iota: k[\bb{A}^{m}] \otimes_{k[\bb{A}^{m}]} (q_{1},...,q_{m}) \hookrightarrow k[\bb{A}^{m}]$ is the inclusion as an ideal; we see the middle map corresponds exactly to the inclusion of $(1 \otimes q_{1},...,1 \otimes q_{m})$ as an ideal. Note that $k[V] \otimes_{k} k[\bb{A}^{m}]$ is a free right $k[\bb{A}^{m}]$ module indexed over a $k$-basis of $k[V]$ and is thus flat, so running the same argument as above with the sequence $0 \to I(W) \to k[\bb{A}^{m}] \to k[W] \to 0$ and the functor $k[V] \otimes_{k} k[\bb{A}^{m}] \otimes_{k[\bb{A}^{m}]} - : k[\bb{A}^{n}]-Mod \to k[\bb{A}^{n}]\otimes_{k}k[\bb{A}^{m}] - Mod$ yields that $k[V \times W] = k[V] \otimes_{k} k[\bb{A}^{m}]/([1 \otimes q_{1}],...,[1\otimes q_{m}]) = k[V] \otimes_{k} k[W]$, proving the claim.  
  \\
  
  b) Assuming $k$ algebraically closed, if $V$ and $W$ are disjoint algebraic sets, we have that $I(V) + I(W) =  I(V \cap W) =  I(\emptyset) = k[x_{1},...,x_{n}]$, and we have that $I(V \cup W) = I(V) \cup I(W)$. We have that by Chinese Remainder Theorem (as $I(V), I(W)$ are comaximal), $k[V \cup W] = k[x_{1},...,x_{n}]/I(V)\cap I(W) = k[x_{1},...,x_{n}]/I(V) \times k[x_{1},...,x_{n}]/I(W) = k[V] \times k[W]$.
\end{proof}
\end{document}