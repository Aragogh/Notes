\documentclass{article}

\usepackage{fancyhdr}
\usepackage{extramarks}
\usepackage{amsmath}
\usepackage{amsthm}
\usepackage{amssymb}
\usepackage{amsfonts}
\usepackage{tikz}
\usepackage[plain]{algorithm}
\usepackage{algpseudocode}
\usepackage{nameref}
\usepackage{cite}
\usepackage{tikz-cd}
\usepackage{mathrsfs}

\usetikzlibrary{automata,positioning}


\topmargin=-0.45in
\evensidemargin=0in
\oddsidemargin=0in
\textwidth=6.5in
\textheight=9.0in
\headsep=0.25in

\linespread{1.1}

\pagestyle{fancy}
\chead{\hmwkTitle}
\lhead{\hmwkAuthorName}
\rhead{\hmwkClass}
\cfoot{\thepage}

\renewcommand\headrulewidth{0.4pt}
\renewcommand\footrulewidth{0.4pt}
\newcommand{\sur}[1]{\ensuremath{^{\textrm{#1}}}}
\newcommand{\sous}[1]{\ensuremath{_{\textrm{#1}}}}
\newcommand{\Hom}{\text{Hom}}
\newcommand{\Tor}{\text{Tor}}
\newcommand{\Ext}{\text{Ext}}
\newcommand{\bb}[1]{\mathbb{#1}}
\newcommand{\fk}[1]{\mathfrak{#1}}
\newcommand{\iso}{\cong}

\setlength\parindent{0pt}

%c
% Create Problem Sections
%

\newtheorem{lemma}{Lemma}
\newtheorem{exercise}{Exercise}
%
% Homework Details
%   - Title
%   - Due date
%   - Class
%   - Section/Time
%   - Instructor
%   - Author
%

\newcommand{\hmwkTitle}{Homework 1}
\newcommand{\hmwkDueDate}{Sep 10th, 2020}
\newcommand{\hmwkClass}{Math 549}
\newcommand{\hmwkClassInstructor}{Professor Julius Ross}
\newcommand{\hmwkAuthorName}{\textbf{Anish Chedalavada}}

%
% Title Page
%

\title{
    \vspace{2in}
    \textmd{\textbf{\hmwkClass:\ \hmwkTitle}}\\
    \vspace{0.1in}
    \textmd{\hmwkDueDate} \\
    \vspace{0.2in}\large{\textit{\hmwkClassInstructor\  }}
    \vspace{2in}
}

\author{\hmwkAuthorName}
\date{}

\begin{document}

\maketitle

\newpage
\begin{exercise}
Lee, Problem 1-7, $p.30$  
\end{exercise}
\begin{proof}
  a) We have that the line from the north pole through any point $x$ on $\bb{S}^{n} \setminus N \subset \bb{R}^{n+1}$ is given by the equation $\lambda_{N}(t) = ((x_{1},...,x_{n+1})- (0,...,1))t + (0,...,1)$. We have that $\lambda_{N}(t)$ crosses the plane $x_{n+1} = 0$ when $t = \frac{1}{1-x_{n+1}}$. This point, which we denote $\sigma'(x)$ is given by the expression:
  \[
    \sigma'(x) = \frac{1}{1-x_{n+1}} \cdot (x_{1},...,x_{n}, 0)
  \]
  Which, upon projection to $\bb{R}^{n}$ by collapsing the $x_{n+1}$ axis, is exactly the expression for the stereographic projection from the north pole. Similarly, the expression for the line from the south pole is given by $\lambda_{S}(t) = ((x_{1},...,x_{n+1}) - (0,...,-1))t + (0,...,-1)$, which crosses the above plane when $t = \frac{1}{1+x_{n+1}}$. We denote this point by $\widetilde{\sigma}'(x)$ and it is given by:
  \[
   \widetilde{\sigma}'(x) = \frac{1}{1+x_{n+1}} \cdot (x_{1},...,x_{n}, 0) = - \frac{1}{1 - (-x_{n+1})} \cdot (-x_{1},...,-x_{n}, 0)
 \]
 Which upon projection is exactly the expression $-\sigma(-x) = \widetilde{\sigma}$. \\

 b) We will show that $\sigma$ is a surjection by showing that it has a right inverse, given by:
 \[
   \sigma^{-1}(u_{1},...,u_{n}) = \frac{1}{|u|^{2}+1}\cdot (2u_{1},...,2u_{n},|u|^{2}-1)
 \]
 Let $u \in \bb{R}^{n}$. We have that $(\sigma \circ \sigma^{-1})(u)$ is given by:
 \begin{align*}
   \frac{1}{1 - \frac{|u|^{2} - 1}{|u|^{2}+1}}\cdot \frac{1}{|u|^{2} + 1} \cdot (2u_{1},...,2u_{n})  &  = \frac{|u|^{2}+1}{|u|^{2}+1 - |u|^{2} + 1}\cdot \frac{1}{|u|^{2} + 1} \cdot (2u_{1},...,2u_{n}) \\
   & = \frac{1}{2} \cdot (2u_{1},...,2u_{n}) = u
 \end{align*}
 Thus, any point $u$ has an element $\sigma^{-1}(u)$ in its preimage. Now we show that $\sigma$ is injective, which will show that the right inverse is also a left inverse as the point $\sigma^{-1}(u)$ is the only point in the preimage of $u$. Given any line from a point $(u,0) \in \bb{R}^{n} \times \{0\} \subset \bb{R}^{n+1}$ to the north pole $(0,...,1)$, we have that the line is given by $\lambda(t) = ((u_{1},...,u_{n},0) - (0,...,1))t + (0,...,1)$. If $a \in \bb{R}$ such that $|\lambda(a)| = 1$, we have that:
 \begin{align*}
   u_{1}^{2}a^{2}+...+u_{n}^{2}a^{2} + (1 - a)^{2} = 1 & \implies a^{2}(u_{1}^{2}+...+u_{n}^{2}) - 2a + 1 = 1 \\
   & \implies a^{2}|u|^{2} = 2a \implies a = \frac{|u|}{2}^{2}
 \end{align*}
 In particular, there is only one point $a$ such that $\lambda(a) \in \bb{S}^{n}$. Thus, by the results of part a) there is only one point $x$ in the fiber of $u$ such that $\sigma(x) = u$. This yields injectivity, and thus the claim.\\

 c) In order to compute the transition maps, note that $\widetilde{\sigma}(x) = -\sigma(-x)$. In particular, $\widetilde{\sigma}^{-1}(u) = - \sigma^{-1}(-u)$. We compute $\sigma(-\sigma^{-1}(-u))$ via the expression:
\[
  \frac{1}{1 + \frac{|u|^{2} - 1}{|u|^{2}+1}}\cdot \frac{1}{|u|^{2} + 1} \cdot (2u_{1},...,2u_{n}) = \frac{1}{2|u|^{2}} \cdot (2u_{1},...,u_{n}) 
\]
Which is a smooth-well defined map for all points $u$ such that $|u| \neq 0$ (being a composition of smooth maps in each coordinate); however, $|u| = 0 \implies -\sigma(-u) = (0,...,1)$ the north pole, for which $\sigma$ is not defined. Thus, we have that $\sigma \circ \widetilde{\sigma}^{-1}$ is a smooth map for all points $u \in \widetilde{\sigma}(U)$for $U = \bb{S}^{n} \setminus N$ the domain of $\sigma$. The other direction ($\widetilde{\sigma} \circ \sigma^{-1})$ is similar, thus yielding that the charts $(\bb{S}^{n} \setminus N, \sigma), (\bb{S}^{n} \setminus S, \widetilde{\sigma})$ are compatible. 
\end{proof}

\begin{exercise}
  Let $M$ be a smooth manifold of dimension $n \geq 1$. Show that $M$ has uncountably many distinct smooth structures on it.
\end{exercise}
\begin{proof}
  We will assume the fact that there exists a locally finite cover of $M$ by compatible charts due to second countability. \\
  \underline{Claim:} There exists a smooth atlas on $M$ such that $\exists \ x\in M$ with $x$ contained in only one chart in the cover. Assume not, then let $n>1$ be minimal such that any $x \in M$ is contained in at least $n$ charts in the cover (can do this by locally finite). Let $x \in M$ such that $x$ is contained in only $n$ charts in the cover. We may remove one of the charts in which $x$ is contained, and this still yields a cover of $M$ by compatible charts as now every point in $M$ is contained in at least $n-1 \geq 1$ charts, and $x \in M$ is contained in only $n-1$ charts, violating the minimality of $n$. Thus, the claim holds. \\
  Now given a smooth atlas $\mathcal{A}$ on $M$ such that the claim holds, let $x \in M$ such that $x$ contained in only one chart $(U,\phi)$ in the atlas. Let $r$ such that $B_{r}(\phi(x)) \subset \phi(U) \subset \bb{R}^{n}$ ($B_{r}$ meaning ball of radius $r$). Define the charts $V_{1}, V_{2} \subset M$ by:
  \[
    V_{1} = \phi^{-1}\left(B_{r}(\phi(x))\right), \ V_{1} = \phi^{-1}\left(\phi(U) - \overline{B_{r/2}(\phi(x))}\right)
    \]
    And $\phi_{1},\phi_{2}$ by restriction of $\phi$. Denote the atlas $\mathcal{A}_{0}$ by replacing $(U,\phi)$ in $\mathcal{A}$ with the charts $(V_{1},\phi_{1}), (V_{2},\phi_{2})$. For $s \geq 1$ denote the collection $\mathcal{A}_{s}$ by replacing $(V_{1},\phi_{1}) \in \mathcal{A}_{0}$ with the chart $(V_{1}, f_{s}\circ \phi_{1})$ with $f_{s}: B_{r}(\phi(x)) \to B_{r}(\phi(x))$ given by the map $f_{s}: y \mapsto \phi(x) + (y-\phi(x))|y-\phi(x)|^{s}$, which is always smooth at all points excluding $y = \phi(x)$, as the composite $|y-\phi(x)|^{s}$ is smooth outside the center, and bijective. As we may extend to a bijection $\widetilde f_{s}$ on the closure of the ball, which is a continous bijection from a compact space to a Hausdorff space, we have that $\widetilde f_{s}$ is a homeomorphism fixing the boundary and thus $f_{s}$ must be as well. The inverse of $f_{s}$ is given by $y \mapsto \phi(x) + (y - \phi(x))|y - \phi(x)|^{\frac{1}{s}-1}$ for $y \neq \phi(x)$, and $\phi(x) \mapsto 0$, which is smooth at all points except at $y = \phi(x)$ (again, composition of smooth functions), as we have the following limit for any $y \to \phi(x)$:
    \[
      \lim_{y\to \phi(x)} \frac{|y - \phi(x)|^{\frac{1}{s}-1}}{|y - \phi(x)|} = \lim_{h\to 0}|h|^{\frac{1}{s}-2} = \infty
    \]
    In particular, we have that for any other chart $(U, \psi) \in \mathcal{A}_{s}$ intersecting $V_{1}$, we have $\psi \circ \phi^{-1} \circ f_{s}^{-1} : B_{r}(\phi(x)) \cap \phi(U\cap V_{1}) \to \bb{R}^{n}$ must be smooth as from the atlas $\mathcal{A}_{0}$ we have that $\psi \circ \phi^{-1}$ is smooth and $f_{s}$ is smooth outside $\phi(x)$ which is not contained in $\phi(U \cap V_{1})$ as we have established that $x \notin U \cap V_{1}$ for any $V_{1} \neq U$. Similarly, we have that $f_{s} \circ \phi \circ \psi^{-1} : \psi(U\cap V_{1}) \to B_{r}(\phi(x))$ must also be smooth, as $\phi \circ \psi^{-1}$ does not map onto $\phi(x)$. Thus, we have that $(V_{1}, f_{s}\circ \phi)$ is compatible with all other charts in $\mathcal{A}_{s}$, and so $\mathcal{A}_{s}$ is an atlas and determines a smooth structure. Now it remains to show that for $t < s$, $\mathcal{A}_{s}$ and $\mathcal{A}_{t}$ do not determine the same smooth structure. Suppose otherwise. Then $\mathcal{A}_{s} \cup \mathcal{A}_{t}$ are together contained in some maximal atlas and all charts are compatible with each other. However, we have that for $(V_{1}, f_{t}\circ \phi), (V_{1}, f_{s}\circ \phi)$, the map
    \[
      f_{t} \circ \phi \circ \phi^{-1} \circ f_{s}^{-1} = f_{t} \circ f_{s}^{-1}: B_{r}(\phi(x)) \to B_{r}(\phi(x))
    \]
    And $f_{t}\circ f_{s}^{-1} (y) = \phi(x) + |y-\phi(x)|^{\frac{t-1}{s}} (y - \phi(x))$ which is not smooth as the difference quotient must diverge as above. Thus, the $\mathcal{A}_{s}$ form a uncountable family of inequivalent atlases, yielding uncountably many distinct smooth structures on $M$.
\end{proof}



\begin{exercise}
  Show that $M = \overline{\bb{B}^{n}} \subset \bb{R}^{n}$ is a manifold-with-boundary in a natural way such that $\partial M = \bb{S}^{n-1}$. Show that the induced smooth structure on Int$(M)$ agrees with the usual smooth structure thought as a subset of $\bb{R}^{n}$, and the induced smooth structure on $\partial M$ agrees with the usual one on $\bb{S}^{n-1}$.  
\end{exercise}
\begin{proof}
  We define two charts on $M$. Define $(U,\phi)$ by the open set $\bb{B}^{n} \subset M \subset \bb{R}^{n}$ embedding via the identity map $\phi = id$, which is open in $\bb{R}^{n}$ and thus in the subspace topology on $M$. Define $V_{1}$ by:
  \[
    V_{1} := \left\{x \in M \ \Big| \ \frac{1}{2} < |x|\right\} \setminus \left\{(0,...,0,x_{n}) \ \Big| \ \frac{1}{2} < x_{n} \leq 1\right\}
  \]
  Let $X = \bb{R}^{n-1} \times [0,1/2)$. Note that $[0,1/2) \iso [0, \infty)$ is a diffeomorphism, so $X \iso \bb{H}^{n}$. Thus, it suffices to define charts going into $X$, as composing with the diffeomorphism to $\bb{H}^{n}$ yields the standard definition of manifoldiwth-boundary. We define $\varphi_{1}: V_{1} \to X$ via:
  \[
    \varphi_{1}: \vec{x} \mapsto \left(\frac{x_{1}}{|\vec{x}| - x_{n}},... \ , \frac{x_{n-1} \cdot |x_{n}|}{|\vec{x}| - x_{n}}, 1 - |\vec{x}| \right)
  \]
  We may may write $V_{1} = \bigcup_{\frac{1}{2} < r \leq 1} \{\bb{S}^{n-1}_{r} \setminus (0,...,0,r)\}$ where $\bb{S}^{n-1}_{r}$ is the $n-1$-sphere of radius $r$. We have that $\varphi_{1}|_{\bb{S}^{n-1}_{r}\setminus (0,...,r)} = \pi(\frac{\vec{x}}{r}) + (0,...,0,1-r)$ where $\pi$ is ordinary stereographic projection. Thus, this map is well-defined (as it is well-defined on each set in the union) bijective onto $X$ (as $\varphi_{1}|_{\bb{S}^{n-1}_{r}\setminus (0,...,r) }: \bb{S}^{n-1}_{r}\setminus (0,...,r) \to \bb{R}^{n-1} \times \{1-r\}$ is bijective) and the transition function $\varphi_{1}|_{\bb{B}^{n} \cap V_{1}} \circ id : \{1/2 < |x| < 1\}$ is smooth as it the composition of functions that are smooth in each coordinate. We have that the inverse of $\varphi_{1}$ is given by:
  \[
    \varphi_{1}^{-1}: (u_{1},...,u_{n-1}) \times r \mapsto \frac{1}{1-r}\cdot\frac{1}{|u|^{2}+1} \cdot(2u_{1},....,2u_{n-1}, |u|^{2}-1)
  \]
  And $id \circ \varphi_{1}^{-1}|_{\bb{R}^{n-1}\times(0,\frac{1}{2})}$ is a smooth map, being the composition of smooth functions in each coordinate. Thus, $(U, id), (V_{1}, \varphi_{1})$ are compatible charts. Analagously to $V_{1}$ we may define $(V_{2}, \varphi_{2})$. Let $\chi_{n}$ be the reflection of $\bb{R}^{n}$ in the $n$th coordinate. Let $V_{2} = \chi_{n}(V_{1})$ and $\varphi_{2} = \varphi_{1} \circ \chi_{n}$. By the same logic as above, $V_{2}$ and $U$ are compatible, and $V_{1}$ and $V_{2}$ are compatible as $\varphi_{2} \circ \varphi_{1}^{-1} = \varphi_{2} \circ id \circ id \circ \varphi_{1}^{-1}$ which is smooth, as it is the composition of smooth functions, and the same logic applies symmetrically. Thus, $(U, id), (V_{1}, \varphi_{1}), (V_{2}, \varphi_{2})$ are compatible and covers $M$, and thus defines a smooth structure on $M$. \\

  Note that all of these charts are open in the subspace topology on $\overline{B^{n}}$ and so the induced topology from the charts agrees with the subspace topology, implying that this chart structure is Hausdorff and second countable. We have that the induced smooth structure on Int$(M)$ is given by $(U, id)$, which is by definition the induced smooth structure on $\bb{B}^{n}$ from $\bb{R}^{n}$. Furthermore, the image of all points in $\bb{R}^{n-1}\times \{0\}$ is all points in $\bb{S}^{n-1}_{1} = \bb{S}^{n-1}$, yielding the first claim. Finally, the induced smooth structure on $\partial M$ is given by $(V_{1}\cap \bb{S}^{n-1}, \varphi_{1}|_{\bb{S}^{n-1}}), (V_{2}\cap \bb{S}^{n-1}, \varphi_{2}|_{\bb{S}^{n-1}})$, which agrees with the structure on $\bb{S}^{n-1}$ from problem 1. 
\end{proof}

\begin{exercise}
  For each of the following maps, compute enough coordinate representations to prove that it is smooth. \\
  a) $p_{n}:\bb{S}^{1} \to \bb{S}^{1}$ given by $p_{n}(z) = z^{n}$ (where we consider $\bb{S}^{1} \iso \bb{C}^{\times}$) the roots of unity.\\
  b) $\alpha : \bb{S}^{n} \to \bb{S}^{n}$ given by $\alpha(x) = -x$.\\
  c) Consider $\bb{S}^{3}$ to be $\{(w,z) \in \bb{C}^{2} \ | \ |w|^{2} + |z|^{2} = 1\}$. Let $F: \bb{S}^{3} \to \bb{S}^{2}$ be given by:
  \[
    F(w,z) = (z\bar{w} + w\bar{z}, iw\bar{z} - iz\bar{w}, z\bar{z} - w\bar{w})
    \]
\end{exercise}
\begin{proof}
  a) For $n =1$ this is just the identity. Assume $n>1$, let $x \in \bb{S}^{1}$ arbitrary. We know that the map $\phi:\bb{R} \to \bb{S}^{n}$ via $\theta \mapsto (\cos\theta, \sin\theta)$ is surjective and continuous, let $k \in \{1,...,8n-1\}$ such that $x \in \phi((\frac{\pi (k-1)}{4n}, \frac{\pi (k+1)}{4n}))$, we will assume known that $\phi_{k} = \phi|_{(\frac{\pi (k-1)}{4n}, \frac{\pi (k+1)}{4n})}$ is a homeomorphism onto its image. Using the stereographic projection coordinates on $\bb{S}^{1}$ from problem 1, and assuming that the image of $\phi_{k}$ is entirely contained in $\bb{S}^{k}\setminus N$, we have that
  \[
    \sigma \circ \phi = \frac{1}{1 - \sin \theta} \cdot (\cos \theta)  
  \]
  Is smooth for $\theta \neq \frac{\pi}{2}$ which we assumed by exclusion of the north pole. Symmetric logic holds for if $Im \phi_{k} \subset \bb{S}^{1} \setminus S$. As $n > 1$, we always have that the domain of $\phi_{k}$ has length less than $\pi$ and so the image cannot contain both the north and south pole (separated by a phase of $\pi$). So we have that $\phi_{k}$ is a smooth homeomorphism onto its image and is thus a local coordinate representation around $x$. Now we have that the map $(p_{n} \circ \phi_{k}): (\frac{\pi (k-1)}{4n}, \frac{\pi (k+1)}{4n}) \to \bb{S}^{1}$ is given by
  \[
    p_{n} \circ \phi_{k}: \theta \mapsto (\cos (n\theta), \sin (n\theta))
  \]
  As the domain of $p_{n} \circ \phi_{k}$ has length less than $\frac{\pi}{2n}$, we have that its image cannot contain both the north and south pole (separated by a phase of $\pi$) and so the image is contained in $\bb{S}^{1} \setminus N$ or $\bb{S}^{1} \setminus S$, and applying stereographic projection as above yields that either the map $\sigma \circ p_{n} \circ \phi_{k}$ or $\widetilde{\sigma} \circ p_{n} \circ \phi_{k}$ is smooth, depending on which point the image excludes. This yields that $p_{n}$ is smooth at any arbitrary $x \in \bb{S}^{1}$. \\

  b) This again follows by stereographic projection. If $x \in \bb{S}^{n} \setminus N$, then we have that $-x \in \bb{S}^{n} \setminus S$, and $x = \sigma^{-1}(u)$ some $\sigma$. We have that $\widetilde{\sigma} \circ \alpha \circ \sigma^{-1}(u) = - \sigma (- (-\sigma^{-1}(u)) = -\sigma (\sigma^{-1}(u)) = -u$, i.e. it is the map $u \mapsto -u$ from $\bb{R}^{n} \to \bb{R}^{n}$, which is smooth. Symmetric logic applies in the other case, yielding that $\alpha$ must be a smooth map.\\

  c) Rewriting the map in terms of real coordinates, where $\bb{C}^{2} \iso \bb{R}^{4}$ as a space and $z = (z_{1},z_{2})$, $w = (w_{1},w_{2})$, we have that $F(z_{1},z_{2},w_{1},w_{2})$ is given by:
  \[
    F(z_{1},z_{2},w_{1},w_{2}) = \left(2(z_{1}w_{1} - z_{2}w_{2}), 2(w_{1}z_{2} + w_{2}z_{1}), z_{1}^{2}+z_{2}^{2} - w_{1}^{2} + w_{2}^{2}\right)
  \]

  Now let $x \in \bb{S}^{3} \setminus N$ any point. We have $u$ associated to $x$ such that $\sigma_{3}^{-1}(u) = x$ the stereographic projection to $\bb{S}^{3}$. Suppose $F: x \mapsto \bb{S}^{2} \setminus N$. We may compute the map $\sigma_{2} \circ F \circ \sigma_{3}^{-1}(u_{1},u_{2},u_{3})$ is given by:
  \begin{align*}
\sigma_{2}\Bigg(2\left(\frac{2u_{1}}{|u|^2 + 1}\frac{2u_{3}}{|u|^2 + 1} - \frac{2u_{2}}{|u|^2 + 1}\frac{|u|^{2} - 1}{|u|^2 + 1}\right), & \ 2\left(\frac{2u_{3}}{|u|^2 + 1}\frac{2u_{2}}{|u|^2 + 1} + \frac{|u|^{2} - 1}{|u|^2 + 1}\frac{2u_{1}}{|u|^2 + 1}\right), \\ & \frac{2u_{1}^{2}}{|u|^2 + 1}+\frac{2u_{2}^{2}}{|u|^2 + 1} - \frac{2u_{3}^{2}}{|u|^2 + 1} + \frac{(|u|^{2} - 1)^{2}}{|u|^2 + 1}\Bigg)
  \end{align*}
  Now the map $\sigma$ is smooth as long as the last coordinate is not equal to $1$, and as $x \mapsto \bb{S}^{n} \setminus N$ we may select neighborhood $u \in N$ sufficiently small such that $F(\sigma_{3}^{-1}(u))$ does not contain the north pole. Thus, as this map is a map composed of functions that are smooth in a neighborhood of $u$ in each coordinate, we have that the map $\sigma_{2} \circ F \circ \sigma_{3}^{-1}$ is smooth in a nieghborhood of of $u$. Symmetric logic applies for if $x$ is the south pole or the image of $x$ is the south pole, yielding that for any arbitrary point $x$, the map $F$ must be smooth. 
\end{proof}

\begin{exercise}
  Let $P: \bb{R}^{n+1}\setminus \{0\} \to \bb{R}^{k+1}\setminus \{0\}$ be a smooth function and $\exists \ d \in \bb{Z}$  such that $\forall \lambda \in \bb{R}, \ x \in \bb{R}^{n+1}$ w.h.t. $P(\lambda x) = \lambda^{d}P(x)$. Show that $P$ induces a smooth map $\widetilde{P}: \bb{RP}^{n} \to \bb{RP}^{k}$.
\end{exercise}
\begin{proof}
  Consider the following diagram:
  \[
    \begin{tikzcd}
\mathbb{R}^{n+1}\setminus \{0\} \arrow[r, "P"] \arrow[d, "\pi_1"'] \arrow[rd, "\pi_2 \circ P" description] & \mathbb{R}^{k+1}\setminus \{0\} \arrow[d, "\pi_2"] \\
\mathbb{RP}^{n} \arrow[r, "\widetilde{P}"', dotted]                                         & \mathbb{RP}^{k}                    
\end{tikzcd}
\]
Where $\pi_{1}, \pi_{2}$ are the canonical quotient maps from Euclidean space to projective space. Note that we have that for any $\lambda \in \bb{R} \setminus \{0\}$, $\pi_{2} \circ P(\lambda x) = \pi_{2}(\lambda^{d}P(x)) = \pi_{2}(P(x))$. In particular, we have that $\pi_{2} \circ P$ factors through the quotient $\bb{R}^{n+1}\setminus \{0\}/\langle x \sim \lambda x, \ \forall \lambda \in \bb{R}\setminus \{0\}\rangle$, in particular it factors through the quotient map $\pi_{1}: \bb{R}^{n+1} \to \bb{RP}^{n}$. The induced map $\widetilde{P}$ is given by $\widetilde{P}([x]) = [P(x)]$ (by definition, as $\pi_{2}\circ P = \widetilde{P} \circ \pi_{1}$). It is well-defined by the rationale above. 

Now to show smoothness. Let $x  \in \bb{R}^{n+1}\setminus \{0\}$. Let $j$ such that $x_{j} \neq 0$ and $i$ arbitrary such that $P_{i}(x) \neq 0$ the $i$th coordinate of $P(x)$. Consider $[x] = [\frac{x_{0}}{x_{j}}:...:1:...:\frac{x_{n}}{x_{j}}] \in \bb{RP}^{n}$. We have that $\widetilde{P}([x]) = [\frac{P_{0}(\overline{x})}{P_{i}(\overline{x})}:...:\frac{P_{k}(\overline{x})}{P_{i}(\overline{x})}] \in \bb{RP}^{k}$ for $\overline{x} = (\frac{x_{0}}{x_{j}},...,1,...,\frac{x_{n}}{x_{j}}) \in \bb{R}^{n}\setminus \{0\}$. We may select $r > 0$ such that for the open cube of side $r$ with $x$ as the center (i.e. the midpoint of every interval) given by $C_{r}(\overline{x}) \subset \bb{R}^{n+1} \setminus \{0\}$ we have that $\forall y \in C_{r}(\overline{x}), \ y_{j} \neq 0$ for some $j$ and $P_{i}(y)$ is nonzero. Now under the standard smooth structure on $\bb{RP}^{n}$, we have that the set of lines such that $\{x_{j} \neq 0\}$ is an open chart homeomorphic to $\bb{R}^{n}$ via the structure map $\phi_{j}: (x_{1},...,x_{n}) \mapsto [x_{1}:...:1:....:x_{n}]$ with $1$ in the $x_{j}$ place. Consider the open cube of side $r$ centered on $\widetilde{x} = (\frac{x_{0}}{x_{j}},...,\hat{\frac{x_{j}}{x_{j}}},...,\frac{x_{n}}{x_{j}}) \in \bb{R}^{n}\setminus \{0\}$ (i.e. $x_{j}$ coordinate omitted). We have that for any $y \in C_{r}(\widetilde{x})$ that $\widetilde{P}(\phi_{j}(y))$ has nonvanishing $i$th coordinate by the fact that $\phi_{j}^{-1}(\widetilde{x}) = [x]$. Let $\phi'_{i}:\bb{R}^{k} \to \{x_{i} \neq 0\} \subset \bb{RP}^{k}$ be the structure map associated to that chart. Then $(\phi'^{-1}_{i} \circ \widetilde{P} \circ \phi_{j})(y)$ for $y \in C_{r}(\widetilde{x}) \subset \bb{R}^{n} \setminus \{0\}$ is given by:
\[
 (y_{0},...,y_{n}) \mapsto \left(\frac{P_{0}(y_{0},...,1,...,y_{n})}{P_{i}(y_{0},...,1,...,y_{n})},...,\frac{P_{n}(y_{0},...,1,...,y_{n})}{P_{i}(y_{0},...,1,...,y_{n})} \right) \text{ with the entry corresponding to $P_{i}$ omitted}.
\]
This is a smooth map from $C_{r}(\widetilde{x}) \to \bb{R}^{n} \setminus \{0\}$, yielding that $\phi'^{-1}_{i} \circ \widetilde{P} \circ \phi_{j}$ is smooth at $[x]$ for any arbitrary point $x \in \bb{R}^{n+1} \setminus \{0\}$ and $j$ such that $x_{j} \neq 0$ and $i$ arbitrary such that $P_{i}(x) \neq 0$. However the set of $j$ such that $x_{j} \neq 0$ and the set of $i$ such that $P_{i}(x) \neq 0$ is precisely the collection of $i,j$ such that $x$ is in the image of the chart $\phi_{j}$ and $P_{i}(x)$ is in the image of $\phi_{i}'$. In particular, we have shown that $\phi'^{-1}_{i} \circ \widetilde{P} \circ \phi_{j}$ is smooth at $[x]$ for all $j$ such that $[x] \in Im(\phi_{j})$, $\widetilde{P}([x]) \in Im(\phi_{i}')$, which is precisely the definition for $\widetilde{P}$ being smooth at $[x]$. This yields the claim. 
\end{proof}
\end{document}