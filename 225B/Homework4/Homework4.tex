\documentclass{article}

\usepackage{fancyhdr}
\usepackage{extramarks}
\usepackage{amsmath}
\usepackage{amsthm}
\usepackage{amssymb}
\usepackage{amsfonts}
\usepackage{tikz}
\usepackage[plain]{algorithm}
\usepackage{algpseudocode}
\usepackage{nameref}
\usepackage{cite}
\usepackage{tikz-cd}
\usepackage{mathrsfs}

\usetikzlibrary{automata,positioning}


\topmargin=-0.45in
\evensidemargin=0in
\oddsidemargin=0in
\textwidth=6.5in
\textheight=9.0in
\headsep=0.25in

\linespread{1.1}

\pagestyle{fancy}
\chead{\hmwkTitle}
\lhead{\hmwkAuthorName}
\rhead{\hmwkClass}
\cfoot{\thepage}

\renewcommand\headrulewidth{0.4pt}
\renewcommand\footrulewidth{0.4pt}
\newcommand{\sur}[1]{\ensuremath{^{\textrm{#1}}}}
\newcommand{\sous}[1]{\ensuremath{_{\textrm{#1}}}}
\newcommand{\Hom}{\text{Hom}}
\newcommand{\st}{\text \ \text{s.t.} \hspace{0.025in} }
\newcommand{\Claim}[1]{\vspace{0.05in} \\ \underline{\textbf{Claim:}} #1 \vspace{0.05in} \\}

\setlength\parindent{0pt}

%c
% Create Problem Sections
%

\newtheorem{lemma}{Lemma}
\newtheorem{exercise}{Exercise}
%
% Homework Details
%   - Title
%   - Due date
%   - Class
%   - Section/Time
%   - Instructor
%   - Author
%

\newcommand{\hmwkTitle}{Homework 4}
\newcommand{\hmwkDueDate}{Feb 8th, 2019}
\newcommand{\hmwkClass}{Math 225B Differential Geometry}
\newcommand{\hmwkClassInstructor}{Professor Peter Petersen}
\newcommand{\hmwkAuthorName}{\textbf{Anish Chedalavada}}

%
% Title Page
%

\title{
    \vspace{2in}
    \textmd{\textbf{\hmwkClass:\ \hmwkTitle}}\\
    \vspace{0.1in}
    \textmd{\hmwkDueDate} \\
    \vspace{0.2in}\large{\textit{\hmwkClassInstructor\  }}
    \vspace{2in}
}

\author{\hmwkAuthorName}
\date{}

\begin{document}

\maketitle

\newpage

\begin{exercise}[Problem 10]
\end{exercise}
\begin{proof}
  \begin{flalign*} a.i) \ \ & \lim_{h \to 0} \frac{1}{h}[\Phi_{t}^{*}(f\cdot w |_{\Phi_{t}}) - f \cdot w)]  &=& \lim_{h \to 0} \frac{1}{h} \ [f(\Phi_{t}(p)) \cdot \Phi_{t}^{*}(w |_{\Phi_{t}}) - f(\Phi_{t}(p)) \cdot w (p)) - f(\Phi_{t}(p) \cdot w(p)) - f(p) \cdot w(p)] \\ & &=& \ (D_{X}f) \cdot w + (L_{x}w) \cdot f \end{flalign*}. 
  \begin{flalign*} & a.ii)  \ \  (L_{X}(w))(Y) = \lim_{t \to 0} \frac{1}{t}[\Phi_{t}^{*}(w |_{\Phi_{t}})(Y) - w(Y)] =  \lim_{t \to 0} \frac{1}{t}[w |_{\Phi_{t}}(D\Phi^{t}Y) - w(Y)] & \\
    & = \lim_{t\to 0} \frac{1}{t}[w |_{\Phi_{t}}(Y|_{\Phi^{t}}) - w(tD\Phi_{t}(L_{X}Y)) - w(Y)] = \lim_{t\to 0} \frac{1}{t}[w(Y)\circ \Phi^{t} - w(tD\Phi_{t}(L_{X}Y)) - w(Y)] & \\
    & = \lim_{t\to 0} \frac{1}{h}[w(Y) + tD_{X}(w(Y)) - w(tD\Phi_{t}(L_{X}Y)) - w(Y)] = D_{X}w(Y) - w(L_{X}Y)& \\
    & \implies D_{X}w(Y) = (L_{X}w)(Y) + w(L_{X}Y)
  \end{flalign*}

  b) The new definition would yield the following changes in sign:
  \[
    L_{X}(f \cdot W) = f \cdot L_{X}Y -Xf \cdot Y \\
    L_{X}(w(Y)) = (L_{X}w)(Y) - w(L_{X}Y) 
  \]
  For parts 4) and 5). 
\end{proof}

\begin{exercise}[Problem 11]
\end{exercise}

\begin{proof}
  a) $\phi^{*}(df)(Y) = (d(f \circ \phi))(Y) = Y(f \circ \phi )$ \\ \\
  b) $[L_{X}df(p)](Y_{p}) = \lim_{t \to 0}\frac{1}{t}[\Phi_{t}^{*}(df|_{\Phi^{t}(p)}) - (df)(p)](Y_{p}) = \lim_{t \to 0}\frac{1}{t}[Y_{p}(f \circ \Phi_{t}) - Y(f)] = \\ \lim_{t \to 0}\frac{1}{t}[Y_{p}[(f \circ \Phi^{t} - f)] = Y_{p}(L_{X}f)$. Thus, $L_{X}df = Y_{p}(L_{X}f) = d(L_{X}f)$. \\ \\
  c) $D_{1}\alpha(0,0) = \lim_{t \to 0}\frac{1}{t}[Y(f \circ \Phi_{-t}) - Y(f)] = - X_{p}(Yf)$ \\
  $D_{2}\alpha(0,0) = \lim_{h \to 0}\frac{1}{h}[Y(f \circ \Phi_{h}) - Yf] = Y_{p}(Xf).$ \\
  Thus, for $c(h) = \alpha(h,h)$ we have $L_{X}Y(p)(f) = -c'(0) = -\alpha'(h,h) = X_{p}(Yf) - Y_{p}(Xf) = [X,Y]f$.
\end{proof}

\begin{exercise}[Problem 13]
\end{exercise}

\begin{proof}
  a) It suffices to show $X, Y, Z$ are linearly independent at any point $p$ as this must then yield a linear isomorphism. Suppose $X_{p} = aY_{p} + bZ_{p}$ for some $a,b \neq 0$. This implies:
  \[
    X_{p} = p_{z}\partial_{y} - p_{y}\partial_{z} = bp_{x}\partial_{y} - bp_{y}\partial_{x} + ap_{x}\partial_{z} - ap_{z}\partial_{x} 
  \]
  Which implies $p_{y} = ap_{x}$, so $p_{z}\partial_{y} = bp_{x}\partial_{y} - bap_{x}\partial_{x} + ap_{z}\partial_{x}$, which means $bp_{x} = p_{z}$ and $-bap_{z} = ap_{z}$, which is only possible if $p = (0,0,0)$ as $a,b \neq 0$. Similar logic shows that $Y_{p}$ and $Z_{p}$ are not linear combinations of the other two, implying they are linearly independent at every point except at $0$, yielding the isomorphism. We compute the Lie Bracket $[X, Y]$ as an illustrative example: we have that \[[X,Y] = (p_{z}\partial_{y} - p_{y}\partial_{z})(p_{x}\partial_{z} - p_{z}\partial_{x}) = p_{z}p_{x}\partial_{y}\partial_{z} - p_{z}p_{z}\partial_{y}\partial_{x} + p_{y}p_{x}\partial_{x}\partial_{z} - p_{y}p_{z}\partial_{z}\partial_{x} = -Z \]
  This agrees with the cross product (1,0,0) $\times$ (0,1,0) = (0,0,-1). Similarly computing the Lie Brackets for other combinations yields the cross product relations, and linearity yields the result on general vector fields. \\ \\
  
  b) Let $\phi$ be the flow along $aX + bY + cZ$. Given an arbitrary point $p \in \mathbb{R}^{3}$, we have that the derivative with respect to the flow at point $p$ is the Lie Bracket for the vector field given by $\vec{p}: p \mapsto \vec{p}$, given by $L_{X}\vec{p} \ = [aX+bY+cZ, p] = (a,b,c) \times (p_{1},p_{2},p_{3})$, the cross product from part $a)$. Thus, the tangent vector of the flow at any point $p$ is always orthogonal to the direction vector $\vec{p}$, which is a rotation.
\end{proof}

\begin{exercise}[Problem 15]
  (Solutions partially adapted from notes by Ian Coley posted on his website at http://www.math.ucla.edu/$\sim$iacoley/hw/diffhwwinter/HW 2.pdf)
\end{exercise}

\begin{proof}
  a) Suppose $D(\partial_{i}) = \sum a_{ij}\partial_{j}$. We may extend $D$ to an operator taking $(k,l)$ tensor fields to themselves via defining $D(w) = \sum -a_{ij} dx^{j}$ for 1-forms, and extending to $(k,l)$-tensors by $D(A \otimes B) = DA \otimes B + A \otimes DB$, tensored over $\mathbb{R}$. In particular, by the tensor product, this is $\mathbb{R}$-linear. We have that $D(CA) = DC(dx^{i_{1}} \otimes ... \otimes dx^{i_{n}} \otimes \partial_{j_{1}} \otimes ... \otimes \partial_{j_{m}}) = \delta_{j_{1}...j_{n}}^{i_{1}...i_{n}}D(B)$ for $B$ some new tensor. Consider $CD(dx^{i_{1}} \otimes ... \otimes dx^{i_{n}} \otimes \partial_{j_{1}} \otimes ... \otimes \partial_{j_{m}}) = \delta_{j_{1}...j_{n}}^{i_{1}...i_{n}}D(B)$, as $C(\partial_{j} \otimes D(dx^{i})) = C(\partial_{j} \otimes \sum a_{im}dx^{m}) = a_{ij}$, but $C(D(\partial_{j}) \otimes dx^{i}) = -a_{ij}$ so application of the contraction cancels out all terms being contracted, yielding that $C(D(A)) = \delta_{j_{1}...j_{n}}^{i_{1}...i_{n}} = DC(A)$. Uniqueness follows as $D'(dx^{i}) = D(dx^{i})$ via using the contraction to match the coordinates. \\ \\

  b) We have that $D_{A}f = 0$ so it is linear on both functions and on vector fields, and satifies Liebniz rule as:
  \[
    D_{A}(fX) =  A(fX) = fA(X) = fA(X) + 0 \cdot X = fA(X) + D_{A}(f) X 
  \]
  and thus satisfies the hypotheses of $a)$, yielding the result. \\ \\
  c) From our construction in part $a)$ extending to 1-forms, we have that $(D_{A}dx^{i})(p) = - \sum a_{ij}dx^{j}(p) = -A^{*}(p)(dx^{i})$, \\
  d) We have $(fL_{X} - L_{fX})(gY) = fL_{X}(gY) - L_{fX}(gY) = g.(fL_{X} - L_{fX}) - (fL_{X} - L_{fX})(Y)$ as it satifies Liebniz rule in each summand. We have that $(fL_{X}- L_{fX})(g) = 0$ as both summands evaluate to be equal. This has a unique extension to (1,1) tensors satisfying the properties above, and it this extension is equal to $D_{X \otimes df}$ as they are equal evaluated on $\partial_{k}$.
\end{proof}

\end{document}