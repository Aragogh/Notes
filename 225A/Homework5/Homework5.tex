\documentclass{article}

\usepackage{fancyhdr}
\usepackage{extramarks}
\usepackage{amsmath}
\usepackage{amsthm}
\usepackage{amssymb}
\usepackage{amsfonts}
\usepackage{tikz}
\usepackage[plain]{algorithm}
\usepackage{algpseudocode}
\usepackage{nameref}
\usepackage{cite}

\usetikzlibrary{automata,positioning}


\topmargin=-0.45in
\evensidemargin=0in
\oddsidemargin=0in
\textwidth=6.5in
\textheight=9.0in
\headsep=0.25in

\linespread{1.1}

\pagestyle{fancy}
\chead{\hmwkTitle}
\lhead{\hmwkAuthorName}
\rhead{\hmwkClass}
\cfoot{\thepage}

\renewcommand\headrulewidth{0.4pt}
\renewcommand\footrulewidth{0.4pt}
\newcommand{\sur}[1]{\ensuremath{^{\textrm{#1}}}}
\newcommand{\sous}[1]{\ensuremath{_{\textrm{#1}}}}

\setlength\parindent{0pt}

%c
% Create Problem Sections
%

\newtheorem{lemma}{Lemma}
\newtheorem{exercise}{Exercise}
%
% Homework Details
%   - Title
%   - Due date
%   - Class
%   - Section/Time
%   - Instructor
%   - Author
%

\newcommand{\hmwkTitle}{Homework 5}
\newcommand{\hmwkDueDate}{November 2nd, 2018}
\newcommand{\hmwkClass}{Math 225A Differential Topology}
\newcommand{\hmwkClassInstructor}{Professor Peter Petersen}
\newcommand{\hmwkAuthorName}{\textbf{Anish Chedalavada}}

%
% Title Page
%

\title{
    \vspace{2in}
    \textmd{\textbf{\hmwkClass:\ \hmwkTitle}}\\
    \vspace{0.1in}
    \textmd{\hmwkDueDate} \\
    \vspace{0.2in}\large{\textit{\hmwkClassInstructor\  }}
    \vspace{2in}
}

\author{\hmwkAuthorName}
\date{}

\begin{document}

\maketitle

\pagebreak

\begin{exercise}
  Find maps of the solid torus into itself having no fixed points. Where does the proof of the Brouwer theorem fail?
  \end{exercise}

  \begin{proof}
    As the torus is given by $S^{1}\times S^{1}$, we may assume the solid torus may be written as $S^{1} \times D^{2}$, where $D^{2}$ is the closed ball of radius 1 in $\mathbb{R}^{2}$. Let $f:S^{1}\times D^{2}\to S^{1} \times D^{2}$ via $f:(\theta,x) \mapsto (\theta + \pi, x)$ where we parametrize $S^{1}$ by $\mathbb{R}/2\pi\mathbb{Z}$. Assume $x$ is a fixed point of this map. Then we have that $\exists x \in [0,2\pi)$ s.t. $x \equiv x \mod \pi$, which is not possible as $0 \not\equiv 0\mod \pi$. Thus, this map has no fixed points. This is possible as the proof of the Brouwer theorem aassumes that every line from $x$ to $f(x)$ (viewed via the parametrization to Euclidean space) must eventually intersect the boundary of the manifold, resulting in a retract to the boundary. However, for the parametrization of $S^{1} \times D^{2}$, we have that this map yields a line from $(\theta,x)$ to $(\theta + \pi, x)$ lying entirely in $S^{1}\times\{x\}$, which never leaves this subspace and thus never passes through the boundary. Thus, we cannot extend this map to a retract onto the boundary of a compact manifold and the proof fails. 
  \end{proof}

  \begin{exercise}
    Prove that if the entries of an $n \times n$ matrix $A$ are all nonnegative, then $A$ has a real nonnegative eigenvalue. 
  \end{exercise}

  \begin{proof}
    We have that as all the entries of $A$ are nonnegative, for any $x$ s.t. $x_{1},...,x_{n} \geq 0$ we have that for $Ax = b$, $b_{1},...,b_{n} \geq 0$. Thus, we may view the map $f: x \mapsto \frac{Ax}{|Ax|}$ as a smooth map, normalizing the image, as a map $f:S^{n-1} \to S^{n-1}$ mapping the compact submanifold with boundary $M = \{x \in S^{n-1} | x_{1},...,x_{n} \geq 0\}$ to itself. We have that the manifold $M$ is homeomorphic to the closed ball $B^{n-1} \subset \mathbb{R}^{n-1}$. We now prove the following general result: If $f:B^{n-1} \to B^{n-1}$ is a continuous map, $f$ has a fixed point. \newline
    If $f$ is a continuous map from $\mathbb{R}^{n} \to \mathbb{R}^{n}$, we have that in a compact set it can be coordinatewise approximated by polynomials $p_{n}$ s.t. $\forall \epsilon > 0 \ \exists \ N \in \mathbb{N}$ s.t. $|p_{n}(x) - f(x)| < \epsilon \ \forall \ n > N$. As polynomials are smooth, for each map $p_{n}$ we have an associated fixed point $c_{n}$ s.t. the $c_{n}$ form a convergent sequence. We have that as $B^{n-1}$ is compact, the function $|f(x) - x|$ must be bounded from below by some value $c > 0$ (as no fixed points by assumption. However, we have that $\exists p_{n}$ s.t. $|p_{n}(x) - f(x)| < \frac{c}{2}$ for all $x \in B^{n-1}$. Thus, for associated fixed point $c_{n} \in B^{n-1}$, we have that $|c_{n} - f(c_{N})| < \frac{c}{2}$, a contradiction. Thus, we have the lemma.
    Using the lemma, we have a map from a set homeomorphic to $B^{n-1}$ to itself, resulting in the existence of a fixed point. Thus, the map $\frac{Ax}{|Ax|}$ must have a fixed point $x$, or that $Ax = |Ax|x$ for some vector $x \in M$. Thus, $A$ has a positive real eigenvalue. 
  \end{proof}

  \begin{exercise}
    Let Y be a compact submanifold of $\mathbb{R}^{M}$, and let $w \in ]mathbb{R}^{M}$. Show that there exists a closest point $y \in Y$, and $w-y \in N_{y}(Y)$.
    \end{exercise}
    \begin{proof}
      We have that the function $|w-y|$ is continuous and thus must attain a minimum at some point $y \in Y$. Let $c: [0,1] \to Y$ be an arbitrary curve s.t. $c(0.5) = y$ (can select this by taking the straight line through 0 in the local parametrization of $y$, assuming $Y$ is not a 0-manifold in which case the result clearly holds. The function $|c(t) - w|^{2}$ attains a minimum at 0.5, and so taking the derivative must yield 0 at 0.5 (local minimum). Thus, we take the derivative of $|c(t) - w|^{2}$, given by $2(w_{1}-c_{1}(t))c_{1}'(t) + ... + 2(w_{n}-c_{n}(t))c_{n}'(t)\big|_{t = 0.5}$. Thus, at 0.5, we have that $w - c(0.5)$ is perpendicular to $c'(0.5) \in T_{y}(Y)$, and thus $w-y \in N_{y}(Y)$.  
    \end{proof}

    \begin{exercise}
      Prove that if $w \in Y^{\epsilon}$, with $Y$ compact, then $\pi(w)$ is the unique point closest to $w$ in $Y$. 
    \end{exercise}
    \begin{proof}
      We have that we may set $\epsilon$ sufficiently small 
      \end{proof}
\end{document}